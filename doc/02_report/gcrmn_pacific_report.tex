\documentclass[a4paper,12pt]{book}

%%%%% PACKAGES %%%%%%%%%%%%%%%%%%%%%%%%%%%%%%%%%%%%%%%%%%%%%%%%%%%%%%%%%%%%%%%%%%%%%%%%%%%%%%%%%%%%%%%%%%%%%

\usepackage{anyfontsize}
\usepackage[latin9]{inputenc}
\usepackage[a4paper,top=2.5cm,bottom=2.5cm,left=2.5cm,right=2.5cm,marginparwidth=1.75cm]{geometry}
\usepackage{afterpage}
\usepackage{xcolor} % To define colors
\usepackage{fancyhdr} % For footer and header
\usepackage[colorlinks=true,citecolor=colormain,urlcolor=black,linkcolor=black]{hyperref}
\usepackage{titlesec}
\usepackage{graphicx}
\usepackage{caption}
\usepackage[round]{natbib}
\usepackage{tcolorbox} % For color box
\usepackage{enumitem}
\usepackage{parskip}
\usepackage{fdsymbol} % For symbols
\usepackage{booktabs}
\usepackage{tikz} % For schemas
\usetikzlibrary{shapes.multipart} % For multilines on schemas
\usepackage{datetime} % For date format
\usepackage{titletoc} % For table of content personalised
\usepackage{listings} % To include code from external file
\usepackage{subcaption} % To include figures 2x1, 2x2...
\usepackage{fixltx2e}
\usepackage{array}
\usepackage{lastpage}
\usepackage{etaremune}
\usepackage{lipsum}
\usepackage{pifont}
\usepackage{multicol}
\usepackage{colortbl}
\usepackage[round]{natbib}
\setlength{\columnsep}{1cm} % width between columns
\usepackage{xparse}
\usetikzlibrary{positioning,calc}

%%%%% COULEURS %%%%%%%%%%%%%%%%%%%%%%%%%%%%%%%%%%%%%%%%%%%%%%%%%%%%%%%%%%%%%%%%%%%%%%%%%%%%%%%%%%%%%%%%%%%

\definecolor{colormain}{HTML}{A059A0} % Main color

\definecolor{Couleur2}{HTML}{6C7A89} % Couleur secondaire (gris fonc�)
\definecolor{Couleur3}{HTML}{BDC3C7} % Couleur terciaire (gris clair)
\definecolor{Couleur4}{HTML}{ECF0F1} % Couleur de fond en-t�te
\definecolor{Couleur5}{HTML}{24252A}

\definecolor{colortableborder}{HTML}{A059A0}
\definecolor{colortable1}{HTML}{f2e9f2}
\definecolor{colortable2}{HTML}{ffffff}
\definecolor{colortable3}{HTML}{a059a0}

%%%%% STYLES %%%%%%%%%%%%%%%%%%%%%%%%%%%%%%%%%%%%%%%%%%%%%%%%%%%%%%%%%%%%%%%%%%%%%%%%%%%%%%%%%%%%%%%%%%%%%%

\newcolumntype{C}[1]{>{\centering}m{#1}}

% Th�me A - Titres des parties
%\newcommand{\THEMEA}[1]{\vspace{1cm} \textcolor{colormain}{\LARGE{#1}}}
\newcommand{\THEMEA}[1]{\vspace{1cm} \hbox to \hsize{\lower.3em\hbox{\textcolor{colormain}{\LARGE{#1}}}\quad \textcolor{Couleur2}{\hrulefill}} \vspace{0.8cm}}

% Th�me B - Notes, classements et dur�e des activit�s de recherche
\newcommand{\THEMEB}[1]{\textcolor{colormain}{\Large{#1}} \vspace{0.5cm}}

% Set Figure names
\captionsetup[figure]{labelfont={color=colormain}, name={\textsf{\ding{115} Figure}}, labelsep=period, justification=justified}

% Set Table names
\captionsetup[table]{labelfont={color=colormain}, name={\textsf{\ding{116} Table}}, labelsep=period, justification=justified}

%%%%% HEADER & FOOTER %%%%%%%%%%%%%%%%%%%%%%%%%%%%%%%%%%%%%%%%%%%%%%%%%%%%%%%%%%%%%%%%%%%%%%%%%%%%%%%%%%%%%%%

\pagestyle{fancy}
\fancyhf{}
\renewcommand{\headrulewidth}{0pt}

\fancyfoot[LE,RO]{\sffamily\footnotesize \textcolor{colormain}{\thepage}}
\fancyfoot[LO]{\sffamily\footnotesize \textcolor{Couleur2}{Status and Trends of Coral Reefs of the Pacific}}
\fancyfoot[RE]{\sffamily\footnotesize \textcolor{Couleur2}{French Polynesia}}

%%%% OTHERS %%%%%%%%%%%%%%%%%%%%%%%%%%%%%%%%%%%%%%%%%%%%%%%%%%%%%%%%%%%%%%%%%%%%%%%%%%%%%%%%%%%%%%%%%%%%

\graphicspath{ {./../../figs/} }

% Set the font of the document
\renewcommand\familydefault{\sfdefault}

% Delete indentation before a paragraph
\setlength{\parindent}{0pt}

\newcommand\textline[4][t]{%
	\par\smallskip\noindent\parbox[#1]{.5\textwidth}{\raggedright\texttt{}#2}%
	\parbox[#1]{.8\textwidth}{\raggedright #4}\par\smallskip%
}

%%% COMMAND TO INSERT INDIVIDUAL GLOBE MAP %%%%%%%%%%%%%%%%%%%%%%%%

\newcommand\mapeez[1]{
	\begin{tikzpicture}[overlay, remember picture]
		\node[anchor=north west, %anchor is upper left corner of the graphic
		xshift=12cm, %shifting around
		yshift=-12cm] 
		at (current page.north west) %left upper corner of the page
		{\includegraphics[width=4cm]{#1}}; 
	\end{tikzpicture}
}

% Larger cdot 
\makeatletter
\newcommand*\bigcdot{\mathpalette\bigcdot@{.5}}
\newcommand*\bigcdot@[2]{\mathbin{\vcenter{\hbox{\scalebox{#2}{$\m@th#1\bullet$}}}}}
\makeatother

% Increase row height for tables
\renewcommand{\arraystretch}{1.2} 

% To have rotated column headers
\NewDocumentCommand{\rot}{O{60} O{1em} m}{\makebox[#2][l]{\rotatebox{#1}{#3}}}%

% Icon to put for contribution within author contribution table
\newcommand{\contrib}{\textcolor{colortable3}{\ding{108}}}

%%%%% DOCUMENT %%%%%%%%%%%%%%%%%%%%%%%%%%%%%%%%%%%%%%%%%%%%%%%%%%%%%%%%%%%%%%%%%%%%%%%%%%%%%%%%%%%%%%%%%%%%%%%

\begin{document}

\thispagestyle{empty}

\vspace*{2cm}
	
\textbf{GCRMN} \textcolor{colormain}{Global Coral Reef Monitoring Network}

\href{www.gcrmn.net}{www.gcrmn.net}
	
\vspace{0.5cm}

\textbf{ICRI} \textcolor{colormain}{International Coral Reef Initiative}

\href{www.icriforum.org}{www.icriforum.org}

\vspace{5cm}

\textcolor{colortable3}{\textbf{Photograph credits}}

{\footnotesize
\textbf{Front cover}. Hannes Klostermann. Coral reefs in Moorea, French Polynesia.

\textbf{Pages 25 and 26}. Hannes Klostermann. Coral reefs in Moorea, French Polynesia.

\textbf{Pages 77 and 78}. Hannes Klostermann. Coral reefs in Moorea, French Polynesia.

\textbf{Back cover}. Hannes Klostermann. Coral reefs in Moorea, French Polynesia.
}

\vspace{1cm}

\textcolor{colortable3}{\textbf{Disclaimer}}

\begin{flushleft}
{\footnotesize The content of this report is solely the opinions of the authors, contributors and editors and do not constitute a statement of policy, decision, or position on behalf of the participating organizations, including those represented on the cover.}
\end{flushleft}

\vspace{1cm}

\textcolor{colortable3}{\textbf{Citation}}

{\footnotesize
	Wicquart J. and Planes S. (eds.), 2023. Status and Trends of Coral Reefs of the Pacific. Global Coral Reef Monitoring Network and International Coral Reef Initiative. doi.
}

\vspace{1cm}

\begin{tikzpicture}[remember picture,overlay]
	\node[opacity=1,inner sep=0pt] at ($(current page.south west)+(3.5cm,4cm)$){\includegraphics[width=2cm]{../../figs/00_misc/01_logo/by-nc-sa.png}};
\end{tikzpicture}

\newpage

%%%%%%%%%%%%%%%%%%%%%%%%%%%%%%%%%%%%%%%%%%%%%%%%%%%%%%%%%%%%%%%%%%%%%%%%%%%%%%%%%%%%%%%%%%%%%%%%%%%%%%%%%%%%%
%%%%% ACKOWLEDGMENTS, ACRONYMS, DEFINITIONS %%%%%%%%%%%%%%%%%%%%%%%%%%%%%%%%%%%%%%%%%%%%%%%%%%%%%%%%%%%%%%%%%
%%%%%%%%%%%%%%%%%%%%%%%%%%%%%%%%%%%%%%%%%%%%%%%%%%%%%%%%%%%%%%%%%%%%%%%%%%%%%%%%%%%%%%%%%%%%%%%%%%%%%%%%%%%%%

\THEMEA{Contents}

\newpage

\THEMEA{Ackowledgments}

\newpage

\THEMEA{Acronyms}

\begin{tabular}{>{\bfseries}>{\color{colortable3}}r>{\itshape}l}
	AIMS  & Australian Institute of Marine Science          \\
	GCRMN & Global Coral Reef Monitoring Network            \\
	NOAA  & National Oceanic and Atmospheric Administration \\
	ENSO  & El Ni�o Southern Oscillation
\end{tabular}

\newpage

\THEMEA{Definitions}

\begin{tabular}{>{\bfseries}>{\color{colortable3}}rl}
	Site  & A unique combination of latitude and longitude          \\
	Survey & A unique combination of latitude and longitude at a given date (or year)           
\end{tabular}

\clearpage
\newpage

%%%%%%%%%%%%%%%%%%%%%%%%%%%%%%%%%%%%%%%%%%%%%%%%%%%%%%%%%%%%%%%%%%%%%%%%%%%%%%%%%%%%%%%%%%%%%%%%%%%%%%%%%%%%%
%%%%% FOREWORD %%%%%%%%%%%%%%%%%%%%%%%%%%%%%%%%%%%%%%%%%%%%%%%%%%%%%%%%%%%%%%%%%%%%%%%%%%%%%%%%%%%%%%%%%%%%%%
%%%%%%%%%%%%%%%%%%%%%%%%%%%%%%%%%%%%%%%%%%%%%%%%%%%%%%%%%%%%%%%%%%%%%%%%%%%%%%%%%%%%%%%%%%%%%%%%%%%%%%%%%%%%%

\begin{tikzpicture}[remember picture,overlay]
	\node[opacity=1,inner sep=0pt] at ($(current page.north)+(0cm,-2.5cm)$){\includegraphics[width=\paperwidth]{../../figs/00_misc/03_cover/executive-summary.png}};
	\node[circle,minimum size=4cm,inner sep=0pt, fill=white] at ($(current page.north)+(5cm,-5.725cm)$) {\textcolor{white}{\textbf{}}};
	\node[circle,minimum size=3.5cm,inner sep=0pt, fill=Couleur3] at ($(current page.north)+(5cm,-5.75cm)$) {\textcolor{white}{\textbf{}}};
	\node[fill,text centered, minimum width=5.4cm, minimum height=1.2cm, fill=colortable3, inner sep=0pt, opacity=0.9] at ($(current page.north)+(-4cm,-5.12cm)$) {{\fontsize{27pt}{40pt}\selectfont \textcolor{white}{FOREWORD}}};
\end{tikzpicture}

\clearpage
\newpage

%%%%%%%%%%%%%%%%%%%%%%%%%%%%%%%%%%%%%%%%%%%%%%%%%%%%%%%%%%%%%%%%%%%%%%%%%%%%%%%%%%%%%%%%%%%%%%%%%%%%%%%%%%%%%
%%%%% EXECUTIVE SUMMARY %%%%%%%%%%%%%%%%%%%%%%%%%%%%%%%%%%%%%%%%%%%%%%%%%%%%%%%%%%%%%%%%%%%%%%%%%%%%%%%%%%%%%
%%%%%%%%%%%%%%%%%%%%%%%%%%%%%%%%%%%%%%%%%%%%%%%%%%%%%%%%%%%%%%%%%%%%%%%%%%%%%%%%%%%%%%%%%%%%%%%%%%%%%%%%%%%%%

\vspace*{5.25cm}

\thispagestyle{empty}

\tikz[remember picture,overlay] \node[opacity=1,inner sep=0pt] at (current page.center){\includegraphics[width=\paperwidth]{../../figs/00_misc/03_cover/executive-summary.png}};

\begin{tikzpicture}[remember picture,overlay]
	\filldraw[color=colortable3, fill=colortable3, opacity=1] ($(current page.north west)+(0cm,-9cm)$) rectangle ($(current page.north west)+(21cm,-12cm)$) ;
\end{tikzpicture}

{\fontsize{40pt}{40pt}\selectfont \textcolor{white}{Executive Summary}}

\vspace{7.5cm}

\begin{flushright}
	\textbf{Coral reefs in American Samoa}
	
	Credit: Shaun Wolfe
\end{flushright}

\clearpage
\newpage

%%%%%%%%%%%%%%%%%%%%%%%%%%%%%%%%%%%%%%%%%%%%%%%%%%%%%%%%%%%%%%%%%%%%%%%%%%%%%%%%%%%%%%%%%%%%%%%%%%%%%%%%%%%%%
%%%%% PART 1 %%%%%%%%%%%%%%%%%%%%%%%%%%%%%%%%%%%%%%%%%%%%%%%%%%%%%%%%%%%%%%%%%%%%%%%%%%%%%%%%%%%%%%%%%%%%%%%%
%%%%%%%%%%%%%%%%%%%%%%%%%%%%%%%%%%%%%%%%%%%%%%%%%%%%%%%%%%%%%%%%%%%%%%%%%%%%%%%%%%%%%%%%%%%%%%%%%%%%%%%%%%%%%

\begin{figure}[h!]
	\centering
	\includegraphics[width=\linewidth]{../../figs/01_part-1/fig-1.png}
	\caption[Map of Economic Exclusive Zone of territories in the Pacific]{Map of Economic Exclusive Zone (EEZ) of territories in the Pacific.}
	\label{part1:fig1}
\end{figure}

blalala

\newpage

\begin{table}[]
	\caption{Human population in countries and territories}
	\input{../../figs/01_part-1/table-2.tex}
	\label{french-polynesia_tbl-1}
\end{table}

%%%%%%%%%%%%%%%%%%%%%%%%%%%%%%%%%%%%%%%%%%%%%%%%%%%%%%%%%%%%%%%%%%%%%%%%%%%%%%%%%%%%%%%%%%%%%%%%%%%%%%%%%%%%%
%%%%% PART 2 %%%%%%%%%%%%%%%%%%%%%%%%%%%%%%%%%%%%%%%%%%%%%%%%%%%%%%%%%%%%%%%%%%%%%%%%%%%%%%%%%%%%%%%%%%%%%%%%
%%%%%%%%%%%%%%%%%%%%%%%%%%%%%%%%%%%%%%%%%%%%%%%%%%%%%%%%%%%%%%%%%%%%%%%%%%%%%%%%%%%%%%%%%%%%%%%%%%%%%%%%%%%%%

\clearpage
\newpage

\thispagestyle{empty}

\tikz[remember picture,overlay] \node[opacity=1,inner sep=0pt] at (current page.center){\includegraphics[width=\paperwidth,height=\paperheight]{../../figs/00_misc/03_cover/part-2_left.png}};

\begin{tikzpicture}[remember picture,overlay]
	\filldraw[color=colortable3, fill=colortable3, opacity=1] ($(current page.north west)+(5cm,-5cm)$) rectangle ($(current page.north west)+(7cm,-7cm)$) ;
	\node at ($(current page.north west)+(6cm,-4.25cm)$) {{\fontsize{26pt}{40pt}\selectfont \textcolor{white}{PART}}};
	\node at ($(current page.north west)+(6cm,-6cm)$) {{\fontsize{50pt}{40pt}\selectfont \textcolor{white}{\textit{2}}}};
	\node at ($(current page.north west)+(6cm,-8cm)$) {{\fontsize{26pt}{40pt}\selectfont \textcolor{white}{Syntheses for countries and territories}}};
\end{tikzpicture}

\clearpage
\newpage

\thispagestyle{empty}

\tikz[remember picture,overlay] \node[opacity=1,inner sep=0pt] at (current page.center){\includegraphics[width=\paperwidth,height=\paperheight]{../../figs/00_misc/03_cover/part-2_right.png}};

\begin{tikzpicture}[remember picture,overlay]
	\filldraw[color=white, fill=white, opacity=1] ($(current page.north west)+(10cm,0cm)$) rectangle ($(current page.south east)+(0cm,0cm)$) ;
	\node[text width=8cm] at ($(current page.north west)+(15cm,-5cm)$) {With minor omissions, each national report contains eight main sections, which have been formatted as follows:};
	\node[circle,minimum size=0.75cm,inner sep=0pt, fill=colortable3] at ($(current page.north west)+(12cm,-8cm)$) {\textcolor{white}{\textbf{1}}};
	
\end{tikzpicture}

\clearpage
\newpage

%%%%% AMERICAN SAMOA %%%%%%%%%%%%%%%%%%%%%%%%%%%%%%%%%%%%%%%%%%%%%%%%%%%%%%%%%%%%%%%%%%%%%%%%%%%%%%%%%%%%%

{\fontsize{30pt}{40pt}\selectfont \textcolor{colortableborder}{\textit{1.}}}

\vspace{0.3cm}

{\fontsize{40pt}{40pt}\selectfont \textcolor{black}{AMERICAN SAMOA}}

\begin{tikzpicture}[remember picture,overlay]
	\filldraw[color=colortable3, fill=colortable3] ($(current page.north west)+(0cm,-5.2cm)$) rectangle ($(current page.south east)+(-3cm,22.5cm)$) ;
\end{tikzpicture}

\vspace{-0.3cm}

\textcolor{white}{\textbf{AUTHORS}}

\textcolor{white}{Gilles Siu\textsuperscript{\textit{1}}, Yannick Chancerelle\textsuperscript{\textit{1}}, Serge Planes\textsuperscript{\textit{1}}, J�r�my Wicquart\textsuperscript{\textit{2}}}

\vspace{0.4cm}

{\scriptsize \textsuperscript{\textcolor{colormain}{\textit{1}}}Laboratoire d'Excellence CORAIL, CRIOBE-USR 3278 CNRS-EPHE-UPVD, BP 1013, Papetoai, Moorea,\\French Polynesia \textcolor{colormain}{\ding{120}} \textsuperscript{\textcolor{colormain}{\textit{2}}}MAREPOLIS, 68 avenue des Corbi�res, 11490 Portel-Des-Corbi�res, France}

\vspace{3cm}

\mapeez{../../figs/02_part-2/fig-1/american-samoa.png}

\newpage

{\fontsize{30pt}{40pt}\selectfont \textcolor{colortableborder}{\textit{5.}}}

\vspace{0.3cm}

{\fontsize{40pt}{40pt}\selectfont \textcolor{black}{FRENCH POLYNESIA}}

\begin{tikzpicture}[remember picture,overlay]
	\filldraw[color=colortable3, fill=colortable3] ($(current page.north west)+(0cm,-5.2cm)$) rectangle ($(current page.south east)+(-3cm,22.5cm)$) ;
\end{tikzpicture}

\vspace{-0.3cm}

\textcolor{white}{\textbf{AUTHORS}}

\textcolor{white}{Gilles Siu\textsuperscript{\textit{1}}, Yannick Chancerelle\textsuperscript{\textit{1}}, Serge Planes\textsuperscript{\textit{1}}, J�r�my Wicquart\textsuperscript{\textit{2}}}

\vspace{0.4cm}

{\scriptsize \textsuperscript{\textcolor{colormain}{\textit{1}}}Laboratoire d'Excellence CORAIL, CRIOBE-USR 3278 CNRS-EPHE-UPVD, BP 1013, Papetoai, Moorea,\\French Polynesia \textcolor{colormain}{\ding{120}} \textsuperscript{\textcolor{colormain}{\textit{2}}}MAREPOLIS, 68 avenue des Corbi�res, 11490 Portel-Des-Corbi�res, France}

\vspace{3cm}

\mapeez{../../figs/02_part-2/fig-1/french-polynesia.png}

\begin{figure}[h!]
	\centering
	\includegraphics[width=12cm]{02_part-2/fig-2/french-polynesia_annotated}
	\caption[Map of the economic exclusive zone of French Polynesia]{Map of the economic exclusive zone of French Polynesia.}
	\label{chapter1:fig1}
\end{figure}

\newpage

\THEMEA{Introduction}

\begin{multicols}{2}

\textcolor{colormain}{Geographic information}

\vspace{0.1cm}

\arrayrulecolor{colortableborder}

\input{../../figs/02_part-2/tbl-1/french-polynesia.tex}

\vspace{0.5cm}
	
French Polynesia is composed of five archipelagos, namely Tuamotu islands, Austral islands, Society islands, Gambier islands, and Marquesas islands (Figure 1). These archipelagos bring together 118 islands and atolls (REF), whose average altitude is 198 m above sea level. With 4,766,689 km, the economic exclusive zone of French Polynesia represents a surface greater than the entire European Union. French Polynesia is bordered by three territories, with Cook Islands on the West, Kiribati (Line Group) on the North, and Pitcairn on the Southeast.

Frech Polynesia is a French Overseas collectivity. The regional capital of French Polynesia is Papeete, located on the island of Tahiti, in the Society islands archipelago.

French Polynesian coral reefs covers approximately 5,981 km, which represent 9.17 \% of the total coral reef extent of the GCRMN Pacific region, and X \% of the world coral reef extent. Coral reefs of French Polynesia are home to approximately X hard coral species (REF) and Y coral reef fish species (REF).

The total human population of French Polynesia living within 100 km from coral reefs in 2020 was X inhabitants. This population has increased by 31.48 \% between 2000 and 2020. The human population is unevenly distributed over the numerous islands and atolls of the territory. Indeed, most of the population is located within the Society archipelago. Thus, numerous coral reefs of French Polynesia are relatively free from direct human impacts, although they can be threatened by illegal fisheries (REF). 

\end{multicols}

\newpage

\THEMEA{Disturbance regime}

\THEMEB{Thermal regime}

\begin{multicols}{2}
	
The long-term average of SST on coral reefs of French Polynesia between 1985 and 2023 was X�C, with a standard deviation of X�C (Figure 4A). Due to the distance of French Polynesia from the Equator, there is a substantial seasonal pattern on SST, with a difference of X�C between the austral summer (peaking in April)  and the austral winter (peaking in September) (Figure 4C). However, because of the wide latitudinal range of French Polynesia, this seasonality in SST varies between the archipelagos of the territory. SST over coral reefs of French Polynesia have increased by X�C between 1985 and 2023, which correspond to a warming rate of X�C per year and X�C per decade (Figure 4A). This warming rate is lower than the warming rate on coral reefs of the entire Pacific region, and slightly lower than the warming rate of the global ocean (Figure X).

\end{multicols}

\begin{multicols}{2}
	
Between 1985 and 2023, maximum DHW over coral reefs of French Polynesia occurred in 2009, 2017, 2021, and 2022 (Figure 5A). However, on these years, the percentage of coral reefs under positive DHW was low (Figure 5B), suggesting limited heat stress to coral reefs. On the contrary, almost 75 \% of French Polynesian coral reefs were exposed to DHW values between 4 and 8�C-weeks in 2016 (Figure 5B), although the maximum DHW value reached this year was below 10�C-weeks (Figure 5A). This result suggests that the heat stress to coral reefs was widespread in 2016, which has potentially led to a massive coral bleaching and ultimately an important mortality of hard corals in the territory.

However, while the Figure 5B provide a broad picture of the occurrence of heat stress on coral, the distribution of positive DHW over coral reefs of French Polynesia was not homogeneous over the territory (Supplementary Material 2). Overall, over a year, when maximum DHW over coral reefs of the northern part of French Polynesia (i.e. Marquesas archipelago) were high, those on the southern part of the territory (i.e. Austral archipelago) were low, and conversely.

\end{multicols}

\newpage

\THEMEB{Cyclones}

Between 1980 and 2022, a total of 14 tropical storms passed within 100 km from a coral reef of French Polynesia, and of these 10 were characterized by sustained wind speed greater than 100 km.h-1. Due to its proximity to the equator, the Marquesas archipelago was less affected by cyclones than the rest of the territory. The cyclone with the highest sustained wind speed over the studied period was the cyclone Veena, in 1983, which passed from 14 km from a coral reef with sustained wind speed of 185 km.h-1.

\begin{figure}[h!]
	\centering
	\includegraphics[width=\textwidth]{02_part-2/fig-5/french-polynesia}
	\caption[Maximum sustained wind speed of cyclones within 100 km of a coral reef between 1980 and 2023 over the territory of French Polynesia.]{Maximum sustained wind speed of cyclones within 100 km of a coral reef between 1980 and 2023 over the territory of French Polynesia. Colors correspond to the cyclone category on the Saffir-Simpson scale. The values of sustained wind speed are extracted from the nearest tropical storms position from a coral reef. For this reason, some sustained wind speed values are below the lower threshold of category 1 Saffir-Simpson scale (i.e. 119 km/h).}
	\label{chapter1:fig2}
\end{figure}

The impacts of some of these cyclones on coral reefs have been described in the scientific literature. Notably the impacts of cyclones occurring between 1982 and 1983 are detailed in (Harmelin-vivien \& Laboute, 1986). More recently, the impact of cyclones Wasa that occurred in 1991 and Oli that occurred in 2010 on hard coral cover has been described by (Lamy et al., 2016).

\newpage

\begin{figure}[h!]
	\centering
	\includegraphics[width=\textwidth]{02_part-2/fig-6/french-polynesia}
	\caption[Maximum sustained wind speed of cyclones within 100 km of a coral reef between 1980 and 2023 over the territory of French Polynesia.]{Maximum sustained wind speed of cyclones within 100 km of a coral reef between 1980 and 2023 over the territory of French Polynesia. Colors correspond to the cyclone category on the Saffir-Simpson scale. The values of sustained wind speed are extracted from the nearest tropical storms position from a coral reef. For this reason, some sustained wind speed values are below the lower threshold of category 1 Saffir-Simpson scale (i.e. 119 km/h).}
	\label{chapter1:fig6}
\end{figure}

\newpage

\THEMEB{Other disturbances}

\newpage

\THEMEA{Benthic cover}

\THEMEB{Spatio-temporal distribution of monitoring}

\begin{multicols}{2}


\textcolor{colormain}{Benthic monitoring information}

\vspace{0.1cm}

\input{../../figs/02_part-2/tbl-2/french-polynesia.tex}

\vspace{0.5cm}

\lipsum[1]

\newpage

\lipsum[1]

{ \centering
	\includegraphics[width=\columnwidth]{02_part-2/fig-3/french-polynesia}\\
	\captionof{figure}{Sea Surface Temperature (SST) grouped per year between 1985 and 2023 over coral reefs of French Polynesia. Each gray line correspond to a unique year and the black line represents the interannual mean SST from 1985 to 2023.}
	\label{french-polynesia_4}
}

\lipsum[3]

\lipsum[3]

\end{multicols}

\newpage

\THEMEB{Benthic cover trends}

\begin{multicols}{2}
	
\lipsum

{ \centering
	\includegraphics[width=\columnwidth]{02_part-2/fig-8/french-polynesia}\\
	\label{french-polynesia_5}
}
	
\end{multicols}

\newpage

%%%%%%%%%%%%%%%%%%%%%%%%%%%%%%%%%%%%%%%%%%%%%%%%%%%%%%%%%%%%%%%%%%%%%%%%%%%%%%%%%%%%%%%%%%%%%%%%%%%%%%%%%%%%%
%%%%% MATERIALS AND METHODS %%%%%%%%%%%%%%%%%%%%%%%%%%%%%%%%%%%%%%%%%%%%%%%%%%%%%%%%%%%%%%%%%%%%%%%%%%%%%%%%%
%%%%%%%%%%%%%%%%%%%%%%%%%%%%%%%%%%%%%%%%%%%%%%%%%%%%%%%%%%%%%%%%%%%%%%%%%%%%%%%%%%%%%%%%%%%%%%%%%%%%%%%%%%%%%

\vspace*{5.25cm}

\thispagestyle{empty}

\tikz[remember picture,overlay] \node[opacity=1,inner sep=0pt] at (current page.center){\includegraphics[width=\paperwidth]{../../figs/00_misc/03_cover/executive-summary.png}};

\begin{tikzpicture}[remember picture,overlay]
	\filldraw[color=colortable3, fill=colortable3, opacity=1] ($(current page.north west)+(0cm,-9cm)$) rectangle ($(current page.north west)+(21cm,-12cm)$) ;
\end{tikzpicture}

{\fontsize{40pt}{40pt}\selectfont \textcolor{white}{Materials and Methods}}

\clearpage
\newpage

\THEMEA{Background maps}

Blalab

\THEMEA{Indicators}

\THEMEB{Maritime area}

\newpage

\begin{table}[]
	\caption{Authors contribution}
	\begin{center}
\begin{tabular}{rlcccccccc}
	 & & \rot{Funding acquisition}
	& \rot{Supervision}
	& \rot{Conceptualization} 
	& \rot{Data acquisition}
	& \rot{Data integration}
	& \rot{Data analysis}
		& \rot{Data visualisation}
			& \rot{Participation to workshop}
	
	
	\\ \hline
		                  \rowcolor{colortable1}
Serge & Planes                         & \contrib & \contrib & &&&&&\\ \hline
\rowcolor{colortable2}
J�r�my & Wicquart                       &                                    & & & &  \contrib &&&\\ \hline
\end{tabular}
	\end{center}
\label{authors-contribution}
\end{table}

\clearpage
\newpage

%%%%%%%%%%%%%%%%%%%%%%%%%%%%%%%%%%%%%%%%%%%%%%%%%%%%%%%%%%%%%%%%%%%%%%%%%%%%%%%%%%%%%%%%%%%%%%%%%%%%%%%%%%%%%
%%%%% ANNEXES %%%%%%%%%%%%%%%%%%%%%%%%%%%%%%%%%%%%%%%%%%%%%%%%%%%%%%%%%%%%%%%%%%%%%%%%%%%%%%%%%%%%%%%%%%%%%%%
%%%%%%%%%%%%%%%%%%%%%%%%%%%%%%%%%%%%%%%%%%%%%%%%%%%%%%%%%%%%%%%%%%%%%%%%%%%%%%%%%%%%%%%%%%%%%%%%%%%%%%%%%%%%%

\vspace*{5.25cm}

\thispagestyle{empty}

\tikz[remember picture,overlay] \node[opacity=1,inner sep=0pt] at (current page.center){\includegraphics[width=\paperwidth]{../../figs/00_misc/03_cover/executive-summary.png}};

\begin{tikzpicture}[remember picture,overlay]
	\filldraw[color=colortable3, fill=colortable3, opacity=1] ($(current page.north west)+(0cm,-9cm)$) rectangle ($(current page.north west)+(21cm,-12cm)$) ;
\end{tikzpicture}

{\fontsize{40pt}{40pt}\selectfont \textcolor{white}{Annexes}}

\vspace{7.5cm}

\begin{flushright}
	\textbf{Coral reefs in American Samoa}
	
	Credit: Shaun Wolfe
\end{flushright}

\clearpage
\newpage

%%%%%%%%%%%%%%%%%%%%%%%%%%%%%%%%%%%%%%%%%%%%%%%%%%%%%%%%%%%%%%%%%%%%%%%%%%%%%%%%%%%%%%%%%%%%%%%%%%%%%%%%%%%%%
%%%%% BIBLIOGRAPHY %%%%%%%%%%%%%%%%%%%%%%%%%%%%%%%%%%%%%%%%%%%%%%%%%%%%%%%%%%%%%%%%%%%%%%%%%%%%%%%%%%%%%%%%%%
%%%%%%%%%%%%%%%%%%%%%%%%%%%%%%%%%%%%%%%%%%%%%%%%%%%%%%%%%%%%%%%%%%%%%%%%%%%%%%%%%%%%%%%%%%%%%%%%%%%%%%%%%%%%%

\vspace*{5.25cm}

\thispagestyle{empty}

\tikz[remember picture,overlay] \node[opacity=1,inner sep=0pt] at (current page.center){\includegraphics[width=\paperwidth]{../../figs/00_misc/03_cover/executive-summary.png}};

\begin{tikzpicture}[remember picture,overlay]
	\filldraw[color=colortable3, fill=colortable3, opacity=1] ($(current page.north west)+(0cm,-9cm)$) rectangle ($(current page.north west)+(21cm,-12cm)$) ;
\end{tikzpicture}

{\fontsize{40pt}{40pt}\selectfont \textcolor{white}{Bibliography}}

\vspace{7.5cm}

\begin{flushright}
	\textbf{Coral reefs in American Samoa}
	
	Credit: Shaun Wolfe
\end{flushright}

\clearpage
\newpage

\end{document}
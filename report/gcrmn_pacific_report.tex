\documentclass[a4paper,12pt,twoside,openany]{book}

%%%%% 1. PACKAGES %%%%%%%%%%%%%%%%%%%%%%%%%%%%%%%%%%%%%%%%%%%%%%%%%%%%%%%%%%%%%%%%%%%%%%%%%%%%%%%%%%%%%%%%%%%%%%%
%%%%%%%%%%%%%%%%%%%%%%%%%%%%%%%%%%%%%%%%%%%%%%%%%%%%%%%%%%%%%%%%%%%%%%%%%%%%%%%%%%%%%%%%%%%%%%%%%%%%%%%%%%%%%%%%%

\usepackage{anyfontsize}
\usepackage[latin9]{inputenc}
\usepackage[a4paper,top=2.5cm,bottom=2.5cm,left=2.5cm,right=2.5cm,marginparwidth=1.75cm]{geometry}
\usepackage{afterpage}
\usepackage{xcolor} % To define colors
\usepackage{fancyhdr} % For footer and header
\usepackage[colorlinks=true,citecolor=firstcolor,urlcolor=firstcolor,linkcolor=firstcolor]{hyperref}
\usepackage[uppercase, sf, explicit, raggedright]{titlesec}
\usepackage{graphicx}
\usepackage{caption}
\usepackage[round]{natbib}
\usepackage{tcolorbox} % For color box
\usepackage{enumitem}
\usepackage{parskip}
\usepackage{fdsymbol} % For symbols
\usepackage{booktabs}
\usepackage{tikz} % For schemas
\usetikzlibrary{shapes.multipart} % For multilines on schemas
\usetikzlibrary{fadings}
\usetikzlibrary{positioning,calc}
\usetikzlibrary{backgrounds}
\usepackage{datetime} % For date format
\usepackage{titletoc} % For table of content personalised
\usepackage{listings} % To include code from external file
\usepackage{subcaption} % To include figures 2x1, 2x2...
\usepackage{fixltx2e}
\usepackage{array}
\usepackage{lastpage}
\usepackage{etaremune}
\usepackage{lipsum}
\usepackage{pifont}
\usepackage{multicol}
\usepackage{colortbl}
\usepackage[round]{natbib}
\setlength{\columnsep}{1cm} % width between columns
\usepackage{xparse}
\usepackage{etoolbox}
\usepackage{bookmark}
\usepackage{changepage}
\usepackage{amsmath}

%\usepackage{draftwatermark}
%\SetWatermarkText{DRAFT} % To add DRAFT watermark

%%%%% 2. COLORS %%%%%%%%%%%%%%%%%%%%%%%%%%%%%%%%%%%%%%%%%%%%%%%%%%%%%%%%%%%%%%%%%%%%%%%%%%%%%%%%%%%%%%%%%%%%%%%%%%%%%
%%%%%%%%%%%%%%%%%%%%%%%%%%%%%%%%%%%%%%%%%%%%%%%%%%%%%%%%%%%%%%%%%%%%%%%%%%%%%%%%%%%%%%%%%%%%%%%%%%%%%%%%%%%%%%%%%%%%%

\definecolor{firstcolor}{HTML}{013C5E}
\definecolor{secondcolor}{HTML}{E4F1FE}
\definecolor{thirdcolor}{HTML}{6C7A89}
\definecolor{fourthcolor}{HTML}{2C3E50}
\definecolor{fifthcolor}{HTML}{ececec}
\definecolor{sixcolor}{HTML}{f8a07e}
\definecolor{sevencolor}{HTML}{ce6693}
\definecolor{eightcolor}{HTML}{a059a0}

\definecolor{fuzzyred}{HTML}{C44D56}

\arrayrulecolor{firstcolor} % Color for table borders

%%%%% 3. HEADER AND FOOTER %%%%%%%%%%%%%%%%%%%%%%%%%%%%%%%%%%%%%%%%%%%%%%%%%%%%%%%%%%%%%%%%%%%%%%%%%%%%%%%%%%%%%%%%%%%
%%%%%%%%%%%%%%%%%%%%%%%%%%%%%%%%%%%%%%%%%%%%%%%%%%%%%%%%%%%%%%%%%%%%%%%%%%%%%%%%%%%%%%%%%%%%%%%%%%%%%%%%%%%%%%%%%%%%%%

\fancyhf{}
\renewcommand{\headrulewidth}{0pt}

\fancypagestyle{plain}{%
	\fancyhf{}%
	\fancyfoot[LE,RO]{\sffamily\footnotesize \textcolor{firstcolor}{\thepage}}
	\fancyfoot[LO]{\sffamily\footnotesize \textcolor{thirdcolor}{Status and Trends of Coral Reefs of the Pacific}}
	\fancyfoot[RE]{\sffamily\footnotesize \textcolor{thirdcolor}{\leftmark}}
    \renewcommand{\chaptermark}[1]{\markboth{\thechapter {##1}}{}}
}

\pagestyle{plain}

%%%%% 4. FIGURES AND TABLES %%%%%%%%%%%%%%%%%%%%%%%%%%%%%%%%%%%%%%%%%%%%%%%%%%%%%%%%%%%%%%%%%%%%%%%%%%%%%%%%%%%%%%%%%%%%%
%%%%%%%%%%%%%%%%%%%%%%%%%%%%%%%%%%%%%%%%%%%%%%%%%%%%%%%%%%%%%%%%%%%%%%%%%%%%%%%%%%%%%%%%%%%%%%%%%%%%%%%%%%%%%%%%%%%%%%%%%

% Set Figure names
\captionsetup[figure]{labelfont={color=firstcolor}, name={\textsf{\ding{115} Figure}}, font=normalsize, labelsep=period, justification=justified}

% Set Table names
\captionsetup[table]{labelfont={color=firstcolor}, name={\textsf{\ding{116} Table}}, font=normalsize, labelsep=period, justification=justified}

% Increase row height for tables
\renewcommand{\arraystretch}{1.2} 

% Include figure in multicol
\newenvironment{figuremulticol}
{\par\medskip\noindent\minipage{\linewidth}}
{\endminipage\par\medskip}

% Additional column types
\newcolumntype{L}[1]{>{\raggedright\let\newline\\\arraybackslash\hspace{0pt}}m{#1}}
\newcolumntype{C}[1]{>{\centering\let\newline\\\arraybackslash\hspace{0pt}}m{#1}}
\newcolumntype{R}[1]{>{\raggedleft\let\newline\\\arraybackslash\hspace{0pt}}m{#1}}

%%%%% 5. TABLE OF CONTENTS %%%%%%%%%%%%%%%%%%%%%%%%%%%%%%%%%%%%%%%%%%%%%%%%%%%%%%%%%%%%%%%%%%%%%%%%%%%%%%%%%%%%%%%%%%%%%%
%%%%%%%%%%%%%%%%%%%%%%%%%%%%%%%%%%%%%%%%%%%%%%%%%%%%%%%%%%%%%%%%%%%%%%%%%%%%%%%%%%%%%%%%%%%%%%%%%%%%%%%%%%%%%%%%%%%%%%%%%

\setcounter{tocdepth}{1}

\renewcommand{\thepart}{}
\renewcommand{\thechapter}{}
\renewcommand{\thesection}{}
\renewcommand{\thesubsection}{}

\titlecontents{part}
[-1em]
{\selectfont\bfseries\color{firstcolor}}
{\contentslabel[\thecontentslabel]{1em}}
{}
{\titlerule*[0.75pc]{.}\contentspage} % dots
[\addvspace{1ex}]

\titlecontents{chapter}
[0em]
{\selectfont\color{black}}
{\contentslabel[\thecontentslabel]{1em}}
{}
{\titlerule*[0.75pc]{.}\contentspage} % dots
[\addvspace{1ex}]

%%%%% 6. STYLES %%%%%%%%%%%%%%%%%%%%%%%%%%%%%%%%%%%%%%%%%%%%%%%%%%%%%%%%%%%%%%%%%%%%%%%%%%%%%%%%%%%%%%%%%%%%%%%%%%%%%%%%%%
%%%%%%%%%%%%%%%%%%%%%%%%%%%%%%%%%%%%%%%%%%%%%%%%%%%%%%%%%%%%%%%%%%%%%%%%%%%%%%%%%%%%%%%%%%%%%%%%%%%%%%%%%%%%%%%%%%%%%%%%%%

%%%%% 6.1 PART %%%%%%%%%%%%%%%%%%%%%%%%%%%%%%%%%%%%%%%%%%%%%%%%%%%%%%%%%%%%%%%%%%%%%%%%%%%%%%%%%%%%%%%%%%%%%%%%%%%%%%%%%%%

\titleformat{\part}
{\fontsize{40pt}{40pt}\selectfont}
{}
{0pt}
{\begin{tikzpicture}[remember picture,overlay]
		\node[opacity=1,inner sep=0pt] at (current page.center){\includegraphics[width=\paperwidth,height=\paperheight]{../figs/00_misc/part.png}};
		\node[anchor=west, fill,text centered, fill=firstcolor, opacity=1, inner xsep=0.65cm, inner ysep=0.65cm] at ($(current page.north west)+(2.5cm,-5cm)$) {{\fontsize{50pt}{40pt}\selectfont \textcolor{white}{\textit{1}}}};
		\node[anchor=west] at ($(current page.north west)+(2.15cm,-3.15cm)$) {{\fontsize{26pt}{40pt}\selectfont \textcolor{white}{PART}}};
		\fill[anchor=west, secondcolor, opacity=0.5] ($(current page.north west)+(-1cm,-6.5cm)$) rectangle ($(current page.north west)+(22cm,-10.5cm)$);
		\node[yshift=-3cm] at (current page.north west)
		{\begin{tikzpicture}[remember picture, overlay]
				(\paperwidth,10cm);
				\node[anchor=north west,xshift=2.5cm,yshift=-4cm,rectangle,inner sep=8pt, text width=15cm]
				{\color{firstcolor}\raggedleft#1};
			\end{tikzpicture}
		};
	\end{tikzpicture}
}
\titlespacing*{\part}{0pt}{50pt}{-10pt} % The last argument control space between chapter title and text

%%%%% 6.2 CHAPTER %%%%%%%%%%%%%%%%%%%%%%%%%%%%%%%%%%%%%%%%%%%%%%%%%%%%%%%%%%%%%%%%%%%%%%%%%%%%%%%%%%%%%%%%%%%%%%%%%%%%%%%%

\makeatletter
\patchcmd{\chapter}{\if@openright\cleardoublepage\else\clearpage\fi}{}{}{}
\makeatother

\titleformat{\chapter}
{\fontsize{20pt}{40pt}\selectfont}
{}
{0pt}
{\begin{tikzpicture}[remember picture,overlay]
		\node[yshift=-3cm] at (current page.north west)
		{\begin{tikzpicture}[remember picture, overlay]
				(\paperwidth,3cm);
				\node[anchor=north west,text centered,xshift=2.5cm,yshift=-0.4cm,rectangle,inner sep=8pt, 
				fill=firstcolor,draw=white, ultra thick]
				{\color{white}\uppercase{\filright#1}};
			\end{tikzpicture}
		};
	\end{tikzpicture}
}
\titlespacing*{\chapter}{0pt}{50pt}{-10pt} % The last argument control space between chapter title and text

%%%%% 6.3 SECTION %%%%%%%%%%%%%%%%%%%%%%%%%%%%%%%%%%%%%%%%%%%%%%%%%%%%%%%%%%%%%%%%%%%%%%%%%%%%%%%%%%%%%%%%%%%%%%%%%%%%%%%%

\titleformat{\section}
{\fontsize{15pt}{20pt}\selectfont\color{firstcolor}}
{}
{0pt}
{\filright#1}

%%%%% 6.4 SUBSECTION %%%%%%%%%%%%%%%%%%%%%%%%%%%%%%%%%%%%%%%%%%%%%%%%%%%%%%%%%%%%%%%%%%%%%%%%%%%%%%%%%%%%%%%%%%%%%%%%%%%%%

\titleformat{\subsection}
{\fontsize{13pt}{20pt}\selectfont\color{firstcolor}}
{}
{0pt}
{#1}

%%%%% 7. FIGURE CAPTIONS FOR PART 2 %%%%%%%%%%%%%%%%%%%%%%%%%%%%%%%%%%%%%%%%%%%%%%%%%%%%%%%%%%%%%%%%%%%%%%%%%%%%%%%%%%%%%%
%%%%%%%%%%%%%%%%%%%%%%%%%%%%%%%%%%%%%%%%%%%%%%%%%%%%%%%%%%%%%%%%%%%%%%%%%%%%%%%%%%%%%%%%%%%%%%%%%%%%%%%%%%%%%%%%%%%%%%%%%%

\newcommand{\titleFigureA}[1]{Sea Surface Temperature (SST) over coral reefs of #1 between 1985 and 2023. Each line, colored by its associated decade, correspond to a unique year. The black line represents the interannual average SST from 1985 to 2023.}

\newcommand{\titleFigureB}[1]{Average Sea Surface Temperature (SST) anomaly between 1985 and 2023 over coral reefs of #1. The black dashed line represents a null SST anomaly, values below this line are negative SST anomalies (\ie cooler than average), and values above this line are positive SST anomalies (\ie warmer than average).}

%%%%% 8. OTHERS %%%%%%%%%%%%%%%%%%%%%%%%%%%%%%%%%%%%%%%%%%%%%%%%%%%%%%%%%%%%%%%%%%%%%%%%%%%%%%%%%%%%%%%%%%%%%%%%%%%%%%%%%%
%%%%%%%%%%%%%%%%%%%%%%%%%%%%%%%%%%%%%%%%%%%%%%%%%%%%%%%%%%%%%%%%%%%%%%%%%%%%%%%%%%%%%%%%%%%%%%%%%%%%%%%%%%%%%%%%%%%%%%%%%%

% Set the font of the document
\renewcommand\familydefault{\sfdefault}

% Delete indentation before a paragraph
\setlength{\parindent}{0pt}

% Icon to separate affiliations
\newcommand{\sepaffil}{\textcolor{firstcolor}{\ding{120} }}

% Icon to put for contribution within author contribution table
\newcommand{\contrib}{\textcolor{firstcolor}{\ding{108}}}
\newcommand{\writing}{\textcolor{firstcolor}{W}}
\newcommand{\review}{\textcolor{firstcolor}{R}}
\newcommand{\ie}{\textit{i.e. }}

% To have rotated column headers
\NewDocumentCommand{\rot}{O{60} O{1em} m}{\makebox[#2][l]{\rotatebox{#1}{#3}}}%

% Style for definition
\newcommand{\definition}[1]{\textcolor{firstcolor}{\textbf{#1}}}

%%%%%%%%%%%%%%%%%%%%%%%%%%%%%%%%%%%%%%%%%%%%%%%%%%%%%%%%%%%%%%%%%%%%%%%%%%%%%%%%%%%%%%%%%%%%%%%%%%%%%%%%%%%%%%%%%%%%%%%%%%
%%%%%%%%%%%%%%%%%%%%%%%%%%%%%%%%%%%%%%%%%%%%%%%%%%%%%%%%%%%%%%%%%%%%%%%%%%%%%%%%%%%%%%%%%%%%%%%%%%%%%%%%%%%%%%%%%%%%%%%%%%
%%%%%%%%%%%%%%%%%%%%%%%%%%%%%%%%%%%%%%%%%%%%%%%%%%%%%%%%%%%%%%%%%%%%%%%%%%%%%%%%%%%%%%%%%%%%%%%%%%%%%%%%%%%%%%%%%%%%%%%%%%

\begin{document}
	
% PAGE COVER
%%%%%%%%%%%%%%%%%%%%%%%%%%%%%%%%%%%%%%%%%%%%%%%%%%%%%%%%%%%%%%%%%%%%%%%%%%%%%%%%%%%%%%%%%%%%%%%%%%%%%%%%%%%%%

\thispagestyle{empty}

\begin{tikzpicture}[remember picture,overlay]
	% Photo over the entire page (A4 format)
	\node[rotate=180] at ($(current page.center)+(0cm,0cm)$){\includegraphics[width=\paperwidth]{../figs/00_misc/part.png}};
	
	% White borders around the page
	\fill[anchor=north west, white] (-1.25,-25) -- (-1.25,-18.5) .. controls (5.5, -16) and (10, -21) .. (17.5, -19) -- (17.5,-25);	
	\fill[anchor=north west, firstcolor] (-1.25,-19) -- (-1.25,-18.5) .. controls (5.5, -16) and (10, -21) .. (17.5, -19) -- (17.5, -19) .. controls (10, -21.25) and (5.5, -16.25) .. (-1.25,-19);
	\tikzfading[name=fade,bottom color=transparent!100,top color=transparent!0]
	\fill[anchor=north west,firstcolor,path fading=fade] ($(current page.north west)+(1.5cm,1cm)$) rectangle ($(current page.north west)+(19.5cm,-8cm)$);
	\fill[anchor=north west, white] ($(current page.north west)+(-1cm,1cm)$) rectangle ($(current page.north west)+(21.5cm,-1.5cm)$);
	\fill[anchor=north west, white] ($(current page.north west)+(-1cm,-28.2cm)$) rectangle ($(current page.north west)+(21.5cm,-30cm)$);
	\fill[anchor=north west, white] ($(current page.north west)+(-1cm,1cm)$) rectangle ($(current page.north west)+(1.5cm,-30cm)$);
	\fill[anchor=north west, white] ($(current page.north west)+(19.5cm,1cm)$) rectangle ($(current page.north west)+(21.5cm,-30cm)$);

	% GCRMN logo
	\node[opacity=1,inner sep=0pt] at ($(current page.north west)+(16.25cm,-3.2cm)$){\includegraphics[height=2cm]{../figs/00_misc/logo-gcrmn.png}};
	% ICRI logo
	\node[opacity=1,inner sep=0pt] at ($(current page.north west)+(10.75cm,-3.2cm)$){\includegraphics[height=2cm]{../figs/00_misc/logo-icri.png}};
		
	% Text
	\node[rectangle, inner sep=4pt, fill=firstcolor, opacity=0, text opacity=1, anchor=center] at ($(current page.center)+(0cm,4cm)$){\fontsize{45pt}{40pt}\fontfamily{phv}\selectfont\color{white}
		\textbf{Status and Trends}
	};
	\node[rectangle, inner sep=4pt, fill=firstcolor, opacity=0, text opacity=1, anchor=center] at ($(current page.center)+(0cm,2cm)$){\fontsize{30pt}{40pt}\fontfamily{phv}\selectfont\color{white}
		\textbf{of coral reefs of the}
	};
	\node[rectangle, inner sep=4pt, fill=firstcolor, opacity=0, text opacity=1, anchor=center] at ($(current page.center)+(-3cm,0cm)$){\fontsize{50pt}{40pt}\fontfamily{phv}\selectfont\color{secondcolor}
		\textbf{PACIFIC}
	};
	
	\node[rectangle,inner sep=6pt, fill=firstcolor] at ($(current page.north west)+(15cm,-16cm)$){\huge \textcolor{white}{1980 - 2023}};
	
	\node[align=left, text width=5cm] at ($(current page.north west)+(5.5cm,-2.4cm)$){\scriptsize \textcolor{white}{EDITED BY}};
	\node[align=left, text width=5cm] at ($(current page.north west)+(5.5cm,-3.1cm)$){\textcolor{white}{{\large J�r�my }{\Large Wicquart}}};
	\node[align=left, text width=5cm] at ($(current page.north west)+(5.5cm,-3.8cm)$){\textcolor{white}{{\large Erica K. }{\Large Towle}}};
	\node[align=left, text width=5cm] at ($(current page.north west)+(5.5cm,-4.5cm)$){\textcolor{white}{{\large Serge }{\Large Planes}}};
	
	% Funding logo
	\node[opacity=1,inner sep=0pt, align=left] at ($(current page.north west)+(2.9cm,-26.75cm)$){\includegraphics[width=3.25cm]{../figs/00_misc/outre-mer.png}};
	\node[opacity=1,inner sep=0pt] at ($(current page.north west)+(8cm,-26.75cm)$){\includegraphics[width=3.25cm]{../figs/00_misc/outre-mer.png}};
	\node[opacity=1,inner sep=0pt] at ($(current page.north west)+(15cm,-26.75cm)$){\includegraphics[width=6cm]{../figs/00_misc/logo-monaco.png}};
		
\end{tikzpicture}

\newpage
	
\thispagestyle{empty}

\begin{tikzpicture}[remember picture,overlay]
	\fill[anchor=west, secondcolor] ($(current page.north west)$) rectangle ($(current page.south east)$);
\end{tikzpicture}

\newpage

\thispagestyle{empty}

\begin{tikzpicture}[remember picture,overlay]
	
	% Text
	\node[rectangle, inner sep=4pt, fill=firstcolor, opacity=0, text opacity=1, anchor=center] at ($(current page.center)+(0cm,4cm)$){\fontsize{45pt}{40pt}\fontfamily{phv}\selectfont\color{black}
		\textbf{Status and Trends}
	};
	\node[rectangle, inner sep=4pt, fill=firstcolor, opacity=0, text opacity=1, anchor=center] at ($(current page.center)+(0cm,2cm)$){\fontsize{30pt}{40pt}\fontfamily{phv}\selectfont\color{black}
		\textbf{of coral reefs of the}
	};
	\node[rectangle, inner sep=4pt, fill=firstcolor, opacity=0, text opacity=1, anchor=center] at ($(current page.center)+(-3cm,0cm)$){\fontsize{50pt}{40pt}\fontfamily{phv}\selectfont\color{secondcolor}
		\textbf{PACIFIC}
	};
	
	\node[rectangle,inner sep=6pt, fill=firstcolor] at ($(current page.north west)+(15cm,-16cm)$){\huge \textcolor{white}{1980 - 2023}};
	
	\node[align=left, text width=5cm] at ($(current page.north west)+(5.5cm,-2.4cm)$){\scriptsize \textcolor{black}{EDITED BY}};
	\node[align=left, text width=5cm] at ($(current page.north west)+(5.5cm,-3.1cm)$){\textcolor{black}{{\large J�r�my }{\Large Wicquart}}};
	\node[align=left, text width=5cm] at ($(current page.north west)+(5.5cm,-3.8cm)$){\textcolor{black}{{\large Erica K. }{\Large Towle}}};
	\node[align=left, text width=5cm] at ($(current page.north west)+(5.5cm,-4.5cm)$){\textcolor{black}{{\large Serge }{\Large Planes}}};
	
\end{tikzpicture}

\newpage

% CITATION PAGE
%%%%%%%%%%%%%%%%%%%%%%%%%%%%%%%%%%%%%%%%%%%%%%%%%%%%%%%%%%%%%%%%%%%%%%%%%%%%%%%%%%%%%%%%%%%%%%%%%%%%%%%%%%%%%

\thispagestyle{empty}

\vspace*{2cm}

\textbf{GCRMN} \textcolor{thirdcolor}{Global Coral Reef Monitoring Network}

\href{www.gcrmn.net}{www.gcrmn.net}

\vspace{0.5cm}

\textbf{ICRI} \textcolor{thirdcolor}{International Coral Reef Initiative}

\href{www.icriforum.org}{www.icriforum.org}

\vspace{4cm}

\textcolor{firstcolor}{\textbf{Photograph credits}}

{\footnotesize
	\textbf{Front cover}. Hannes Klostermann. Coral reefs in Moorea, French Polynesia.
	
	\textbf{Page 25}. Hannes Klostermann. Coral reefs in Moorea, French Polynesia.
		
	\textbf{Page 48}. Hannes Klostermann. Coral reefs in Moorea, French Polynesia.
	
	\textbf{Back cover}. Hannes Klostermann. Coral reefs in Moorea, French Polynesia.
}

\vspace{1cm}

\textcolor{firstcolor}{\textbf{Disclaimer}}

\begin{flushleft}
	{\footnotesize The content of this report is solely the opinions of the authors, contributors and editors and do not constitute a statement of policy, decision, or position on behalf of the participating organizations, including those represented on the cover.}
\end{flushleft}

\vspace{1cm}

\textcolor{firstcolor}{\textbf{Citation}}

{\footnotesize
	Wicquart J., Towle E. K., and Planes S. (eds.), 2024. Status and Trends of Coral Reefs of the Pacific. Global Coral Reef Monitoring Network and International Coral Reef Initiative. DOI. ISBN.
}

\vspace{1cm}

\begin{tikzpicture}[remember picture,overlay]
	\node[opacity=1,inner sep=0pt] at ($(current page.south west)+(3.5cm,4cm)$){\includegraphics[width=2cm]{../figs/00_misc/by-nc-sa.png}};
	\node[align=left] at ($(current page.south west)+(8.5cm,3.95cm)$) {\footnotesize Copyright Global Coral Reef Monitoring Network\\ \footnotesize and International Coral Reef Initiative, 2024.};
\end{tikzpicture}

\newpage

% ACKNOWLEDGMENTS
%%%%%%%%%%%%%%%%%%%%%%%%%%%%%%%%%%%%%%%%%%%%%%%%%%%%%%%%%%%%%%%%%%%%%%%%%%%%%%%%%%%%%%%%%%%%%%%%%%%%%%%%%%%%%

\begin{tikzpicture}[remember picture,overlay]
	\node[opacity=1,inner sep=0pt, anchor=north west] at ($(current page.north west)+(0cm,0cm)$){\includegraphics[width=\paperwidth]{../figs/00_misc/chapter.png}};
	\node[anchor=west] at ($(current page.north west)+(2.5cm,-0.75cm)$) {\textcolor{white}{{\scriptsize Hannes Klostermann. Coral reefs in Moorea, French Polynesia}}};
\end{tikzpicture}

\chapter{Acknowledgments}

\begin{multicols}{2}
	
	\lipsum[1-4]
	
\end{multicols}

\newpage

% TABLE OF CONTENTS
%%%%%%%%%%%%%%%%%%%%%%%%%%%%%%%%%%%%%%%%%%%%%%%%%%%%%%%%%%%%%%%%%%%%%%%%%%%%%%%%%%%%%%%%%%%%%%%%%%%%%%%%%%%%%

\begin{tikzpicture}[remember picture,overlay]
	\node[opacity=1,inner sep=0pt, anchor=north west] at ($(current page.north west)+(0cm,0cm)$){\includegraphics[width=\paperwidth]{../figs/00_misc/chapter.png}};
	\node[anchor=west] at ($(current page.north west)+(2.5cm,-0.75cm)$) {\textcolor{white}{{\scriptsize Hannes Klostermann. Coral reefs in Moorea, French Polynesia}}};
\end{tikzpicture}

\fancypagestyle{plain}{%
	\fancyhf{}%
	\fancyfoot[LE,RO]{\sffamily\footnotesize \textcolor{firstcolor}{\thepage}}
	\fancyfoot[LO]{\sffamily\footnotesize \textcolor{thirdcolor}{Status and Trends of Coral Reefs of the Pacific}}
	\fancyfoot[RE]{\sffamily\footnotesize \textcolor{thirdcolor}{Table of contents}}
}

\pagestyle{plain}

\renewcommand{\contentsname}{TABLE OF CONTENTS}
\tableofcontents
\chaptermark{Table of contents}
\addcontentsline{toc}{chapter}{Table of contents}

\newpage

% ACRONYMS
%%%%%%%%%%%%%%%%%%%%%%%%%%%%%%%%%%%%%%%%%%%%%%%%%%%%%%%%%%%%%%%%%%%%%%%%%%%%%%%%%%%%%%%%%%%%%%%%%%%%%%%%%%%%%

\begin{tikzpicture}[remember picture,overlay]
	\node[opacity=1,inner sep=0pt, anchor=north west] at ($(current page.north west)+(0cm,0cm)$){\includegraphics[width=\paperwidth]{../figs/00_misc/chapter.png}};
	\node[anchor=west] at ($(current page.north west)+(2.5cm,-0.75cm)$) {\textcolor{white}{{\scriptsize Hannes Klostermann. Coral reefs in Moorea, French Polynesia}}};
\end{tikzpicture}

\fancypagestyle{plain}{%
	\fancyhf{}%
	\fancyfoot[LE,RO]{\sffamily\footnotesize \textcolor{firstcolor}{\thepage}}
	\fancyfoot[LO]{\sffamily\footnotesize \textcolor{thirdcolor}{Status and Trends of Coral Reefs of the Pacific}}
	\fancyfoot[RE]{\sffamily\footnotesize \textcolor{thirdcolor}{\leftmark}}
}

\pagestyle{plain}

\chapter{Acronyms}

\begin{tabular}{>{\bfseries}>{\color{firstcolor}}rl}
	ACA   & Allen Coral Atlas                               \\
	AIMS  & Australian Institute of Marine Science          \\
	CCA   & Crustose Coralline Algae                        \\
	COTS  & Crown-of-Thorns Starfish                        \\
	DHW   & Degree Heating Week                             \\
	EEZ   & Economic Exclusive Zone                         \\
	ENSO  & El Ni\~{n}o Southern Oscillation                \\
	GCRMN & Global Coral Reef Monitoring Network            \\
	GEE   & Google Earth Engine                             \\
	ICRI  & International Coral Reef Initiative             \\
	IPCC  & Intergovernmental Panel on Climate Change       \\
	ML    & Machine Learning                                \\
	MPA   & Marine Protected Area                           \\
	NOAA  & National Oceanic and Atmospheric Administration \\
	SST   & Sea Surface Temperature                         \\
	WoRMS & World Register of Marine Species
\end{tabular}

\newpage

% DEFINITIONS
%%%%%%%%%%%%%%%%%%%%%%%%%%%%%%%%%%%%%%%%%%%%%%%%%%%%%%%%%%%%%%%%%%%%%%%%%%%%%%%%%%%%%%%%%%%%%%%%%%%%%%%%%%%%%

\begin{tikzpicture}[remember picture,overlay]
	\node[opacity=1,inner sep=0pt, anchor=north west] at ($(current page.north west)+(0cm,0cm)$){\includegraphics[width=\paperwidth]{../figs/00_misc/chapter.png}};
	\node[anchor=west] at ($(current page.north west)+(2.5cm,-0.75cm)$) {\textcolor{white}{{\scriptsize Hannes Klostermann. Coral reefs in Moorea, French Polynesia}}};
\end{tikzpicture}

\chapter{Glossary}

\begin{multicols}{2}
	
	\definition{Dataset}
	
	A collection of related sets of information that is composed of separate elements (data files) but can be manipulated as a unit by a computer.
	
	\definition{Data integration}
	
	Process of combining, merging, or joining data together, in order to make what were distinct, multiple data objects, into a single, unified data object.
	
	\definition{Non-native species}
	
	A species occurring outside its natural past or present range and dispersal potential, whose presence and dispersal is due to intentional or unintentional human action \citep{walther2009alien}.
			
\end{multicols}

\newpage

% FOREWORD
%%%%%%%%%%%%%%%%%%%%%%%%%%%%%%%%%%%%%%%%%%%%%%%%%%%%%%%%%%%%%%%%%%%%%%%%%%%%%%%%%%%%%%%%%%%%%%%%%%%%%%%%%%%%%

\begin{tikzpicture}[remember picture,overlay]
	\node[opacity=1,inner sep=0pt, anchor=north west] at ($(current page.north west)+(0cm,0cm)$){\includegraphics[width=\paperwidth]{../figs/00_misc/chapter.png}};
    \node[circle,minimum size=4cm,inner sep=0pt, fill=thirdcolor, draw=white, ultra thick] at ($(current page.north)+(5cm,-3.25cm)$) {\textcolor{white}{\textbf{}}};
	\node[anchor=west] at ($(current page.north west)+(2.5cm,-0.75cm)$) {\textcolor{white}{{\scriptsize Hannes Klostermann. Coral reefs in Moorea, French Polynesia}}};
    \clip ($(current page.north)+(5cm,-3.25cm)$) circle (1.99cm);
    \node[anchor=west] at ($(current page.north)+(2.5cm,-3.6cm)$) {\includegraphics[width = 5cm]{../figs/00_misc/foreword-photo.jpg}};
\end{tikzpicture}

\chapter{Foreword}

\begin{multicols*}{2}
	
\lipsum[1-4]

\begin{flushright}
	
	\vspace{0.5cm}
	
	\textcolor{firstcolor}{Mr. John Smith}
	
	Head director of UNEP
	
	\vspace{0.25cm}
	
	\includegraphics[width = 3cm]{../figs/00_misc/foreword-signature.png}
\end{flushright}

\end{multicols*}

\newpage

% EXECUTIVE SUMMARY
%%%%%%%%%%%%%%%%%%%%%%%%%%%%%%%%%%%%%%%%%%%%%%%%%%%%%%%%%%%%%%%%%%%%%%%%%%%%%%%%%%%%%%%%%%%%%%%%%%%%%%%%%%%%%

\begin{tikzpicture}[remember picture,overlay]
	\node[opacity=1,inner sep=0pt, anchor=north west] at ($(current page.north west)+(0cm,0cm)$){\includegraphics[width=\paperwidth]{../figs/00_misc/chapter.png}};
	\node[anchor=west] at ($(current page.north west)+(2.5cm,-0.75cm)$) {\textcolor{white}{{\scriptsize Hannes Klostermann. Coral reefs in Moorea, French Polynesia}}};
\end{tikzpicture}

\chapter{Executive summary}

\newcommand*\circled[2]{\tikz[baseline=(char.base)]{
		\node[shape=circle,fill=#1, draw=white,inner sep=2pt] (char) {\textcolor{white}{#2}};}}

\begin{multicols}{2}
	
	{\Large{\textcolor{firstcolor}{Coral reefs of the Pacific}}}
	
	\vspace{0.5cm}
	
	\circled{firstcolor}{1} The extent of coral reefs of the Pacific is 62,000 km which represent 26.5 \% of the extent of the world's coral reefs.
	
	\circled{firstcolor}{2} The extent of coral reefs of the Pacific is 62,000 km which represent 26.5 \% of the extent of the world's coral reefs.
	
	\circled{firstcolor}{3} The extent of coral reefs of the Pacific is 62,000 km which represent 26.5 \% of the extent of the world's coral reefs.
	
	\vspace{0.5cm}
		
	{\Large{\textcolor{sixcolor}{Threats}}}
	
	\vspace{0.5cm}
	
	\circled{sixcolor}{1} The extent of coral reefs of the Pacific is 62,000 km which represent 26.5 \% of the extent of the world's coral reefs.
	
	\circled{sixcolor}{2} The extent of coral reefs of the Pacific is 62,000 km which represent 26.5 \% of the extent of the world's coral reefs.
	
	\circled{sixcolor}{3} The extent of coral reefs of the Pacific is 62,000 km which represent 26.5 \% of the extent of the world's coral reefs.
		
	\vspace{0.5cm}
			
	{\Large{\textcolor{sevencolor}{Trends}}}
	
	\vspace{0.5cm}
	
	\circled{sevencolor}{1} The extent of coral reefs of the Pacific is 62,000 km which represent 26.5 \% of the extent of the world's coral reefs.
	
	\circled{sevencolor}{2} The extent of coral reefs of the Pacific is 62,000 km which represent 26.5 \% of the extent of the world's coral reefs.
	
	\circled{sevencolor}{3} The extent of coral reefs of the Pacific is 62,000 km which represent 26.5 \% of the extent of the world's coral reefs.
		
	\vspace{0.5cm}
		
	{\Large{\textcolor{eightcolor}{Recommandations}}}
	
	\vspace{0.5cm}
	
	\circled{eightcolor}{1} The extent of coral reefs of the Pacific is 62,000 km which represent 26.5 \% of the extent of the world's coral reefs.
	
	\circled{eightcolor}{2} The extent of coral reefs of the Pacific is 62,000 km which represent 26.5 \% of the extent of the world's coral reefs.
	
	\circled{eightcolor}{3} The extent of coral reefs of the Pacific is 62,000 km which represent 26.5 \% of the extent of the world's coral reefs.
		
\end{multicols}

\newpage

% INTRODUCTION
%%%%%%%%%%%%%%%%%%%%%%%%%%%%%%%%%%%%%%%%%%%%%%%%%%%%%%%%%%%%%%%%%%%%%%%%%%%%%%%%%%%%%%%%%%%%%%%%%%%%%%%%%%%%%

\begin{tikzpicture}[remember picture,overlay]
	\node[opacity=1,inner sep=0pt, anchor=north west] at ($(current page.north west)+(0cm,0cm)$){\includegraphics[width=\paperwidth]{../figs/00_misc/chapter.png}};
	\node[anchor=west] at ($(current page.north west)+(2.5cm,-0.75cm)$) {\textcolor{white}{{\scriptsize Hannes Klostermann. Coral reefs in Moorea, French Polynesia}}};
\end{tikzpicture}

\chapter{Introduction}

\begin{multicols}{2}
	
	\section*{The GCRMN: context and objectives}
	
	Coral reefs of the Anthropocene are affected by a wide range of direct and indirect anthropogenic stressors (Wilkinson, 1999). The cummulated impacts of these stressors have affected the state and functionning of coral reefs (REF). For example, multiple studies have shown a decrease in hard coral cover (REF), fish biomass (REF), and accretion (REF) in numerous coral reefs of the world, even on those which have long been considered too remote from human activities to be affected by them (REF).
	
	To reverse the global decline of coral reefs, or at least to stop it, several national and international policies have been undertaken. For example, the Aichi target number 10 from the Convention on Biological Diversity aimed to minimize the anthropogeneic pressures affecting coral reefs to maintain their integrity and functionning (UNEP/CBD/COP/10/27, 2011). The Kunming-Montreal Global Biodiversity Framework reaffirmed the objectives of the Aichi targets for the 2020-2030 decade (CBD/COP/DEC/15/4, 2022). However, to measure target progress of these initiatives, it is crucial to have quantitative syntheses at large spatial scales (Orr et al., 2022).
	
	The Global Coral Reef Monitoring Network (GCRMN) aims to produce regular syntheses about status and trends of coral reefs, to inform policy makers, managers, and scientists. Created in 1995 as an operational network of the International Coral Reef Initiative (ICRI), the GCRMN operates through 10 regional nodes (REF ).  These nodes . Importance des cntributeurs et indicateurs utilis�s.
			
	\section*{History of GCRMN reports}
	
	\lipsum[1]
		
	\section*{The 2018 GCRMN Pacific report}
	
	\lipsum[1]
		
	\section*{Objectives of the current report}
	
	\lipsum[1]
		
	\section*{Structure of the current report}
	
	\lipsum[1]
	
\end{multicols}

%%%%%%%%%%%%%%%%%%%%%%%%%%%%%%%%%%%%%%%%%%%%%%%%%%%%%%%%%%%%%%%%%%%%%%%%%%%%%%%%%%%%%%%%%%%%%%%%%%%%%%%%%%%%%%%%%%%%%%%%%%%%%%%%%%%%%%
%%%%% PART 1 %%%%%%%%%%%%%%%%%%%%%%%%%%%%%%%%%%%%%%%%%%%%%%%%%%%%%%%%%%%%%%%%%%%%%%%%%%%%%%%%%%%%%%%%%%%%%%%%%%%%%%%%%%%%%%%%%%%%%%%%%
%%%%%%%%%%%%%%%%%%%%%%%%%%%%%%%%%%%%%%%%%%%%%%%%%%%%%%%%%%%%%%%%%%%%%%%%%%%%%%%%%%%%%%%%%%%%%%%%%%%%%%%%%%%%%%%%%%%%%%%%%%%%%%%%%%%%%%

\newpage

\newcounter{chaptercounter}
\setcounter{chaptercounter}{1}

\renewcommand{\thefigure}{\arabic{part}.\arabic{chaptercounter}.\arabic{figure}}
\renewcommand{\thetable}{\arabic{part}.\arabic{chaptercounter}.\arabic{table}}

\fancypagestyle{plain}{%
	\fancyhf{}%
	\fancyfoot[RO,LE]{}%
	\renewcommand{\headrulewidth}{0pt}%
}

\pagestyle{plain}

\part[Part 1. Synthesis for the Pacific region]{Synthesis for the Pacific region}

\newpage

\begin{tikzpicture}[remember picture,overlay]
	\fill[anchor=west, secondcolor] ($(current page.north west)$) rectangle ($(current page.south east)$);
\end{tikzpicture}

\newpage

%%%%%%%%%%%%%%%%%%%%%%%%%%%%%%%%%%%%%%%%%%%%%%%%%%%%%%%%%%%%%%%%%%%%%%%%%%%%%%%%%%%%%%%%%%%%%%%%%%%%%%%%%%%%%

%%%%% REDEFINE %%%%%%%

% Footer

\fancypagestyle{plain}{%
	\fancyhf{}%
	\fancyfoot[LE,RO]{\sffamily\footnotesize \textcolor{firstcolor}{\thepage}}
	\fancyfoot[LO]{\sffamily\footnotesize \textcolor{thirdcolor}{Status and Trends of Coral Reefs of the Pacific}}
	\fancyfoot[RE]{\sffamily\footnotesize \textcolor{thirdcolor}{\leftmark}}
}

\pagestyle{plain}

% Redefine TOC appearance for chapters %%

\titlecontents{chapter}
[1.5em]
{\selectfont\color{black}}
{\contentslabel[\thecontentslabel]{1em}}
{}
{\titlerule*[0.75pc]{.}\contentspage} % dots
[\addvspace{1ex}]

% Redefine title appearance for chapters %%

\titleformat{\chapter}
	{\fontsize{20pt}{40pt}\selectfont\color{firstcolor}}
	{}
	{0pt}
	{\filright#1}
	\titlespacing*{\chapter}{0pt}{-35pt}{10pt} % The last argument control space between chapter title and text

%%%%%%%%%%%%%%%%%%%%%

\chapter{Coral reefs of the Pacific}

\begin{multicols}{2}

\section{Distribution and extent}

\lipsum[1]

\end{multicols}

\begin{figure}[h!]
	\centering
	\includegraphics[width=\linewidth]{../figs/01_part-1/fig-1.png}
	\caption{Map of the economic exclusive zones (EEZ) of the Pacific GCRMN region. The unlabelled economic exclusive zone represents the overlapping claimed area of Matthew and Hunter Islands, by France and Vanuatu. Wa.: Wake Island, Jo.: Johnston Atoll, Pa.:Palmyra Atoll, Ja.: Jarvis Island, Na.: Nauru, Sa.: Samoa, Am. Sa.: American Samoa.}
	\label{part1:chap1:fig-1}
\end{figure}

\newpage

\begin{table}[h!]
	\caption{Comparison of the absolute extent of coral reefs between Pacific countries and territories, the relative extent of Pacific reefs and the relative extent of the world's reefs. Since the values for ``Kiribati" and ``Pacific Remote Island Area" rows corresponds to the sum of their associated EEZ, they were not used to calculate the totals.}
	\input{../figs/01_part-1/table-1.tex}
	\label{part1:chap1:tbl-1}
\end{table}

\newpage

\begin{figure}[h]
	\centering
	\includegraphics[width=\linewidth]{../figs/01_part-1/fig-2.png}
	\caption[Relative coral reef extent between GCRMN regions and between countries and territories within the GCRMN Pacific region]{Relative coral reef extent between GCRMN regions (A) and between countries and territories within the GCRMN Pacific region (B). The area of each polygon is proportional to the relative coral reef extent of the associated region. The absolute values of coral reef extent for countries and territories of the Pacific are provided in \autoref{part1:chap1:tbl-1}. In subfigure A, WIO is the acronym for Western Indian Ocean.}
	\label{part1:chap1:fig-2}
\end{figure}

\newpage

\begin{multicols}{2}

\section{Biodiversity}

\lipsum[1-3]

\section{Human population}

\lipsum[1-3]
	
\begin{figuremulticol}
	\centering
	\includegraphics[width=\linewidth]{../figs/01_part-1/fig-3.png}
	\captionof{figure}{Evolution of the human population living within 100 km from a coral reef between 2000 and 2020, for countries and territories of the Pacific region.}
	\label{part1:chap1:fig-3}
\end{figuremulticol}

\lipsum[1]

\end{multicols}

\newpage

\begin{table}[h!]
	\caption{Human population in countries and territories}
	\input{../figs/01_part-1/table-2.tex}
	\label{part1:chap1:tbl-2}
\end{table}

\newpage

\begin{multicols}{2}
	
\section{Ecosystem services}

\lipsum[1-3]

\end{multicols}

\newpage

\chapter{Threats to coral reefs of the Pacific}

\setcounter{chaptercounter}{2}

\begin{multicols}{2}

\section{Thermal regime}

\lipsum[1-3]

\subsection{Long-term change in SST}

\lipsum[1-2]

\begin{figuremulticol}
	\centering
	\includegraphics[width=\linewidth]{../figs/01_part-1/fig-4.png}
	\captionof{figure}{Comparison of average changes in SST of coral reefs between Pacific countries and territories over the period 1980 to 2023. Points at the left (in blue) of the vertical black line correspond to territories with a decrease of SST over the period, whereas points at the right (in red) correspond to territories with an increase in SST. Values indicated near the points are annual warming rate (�C per year). See Table X for values.}
	\label{part1:chap2:fig-4}
\end{figuremulticol}

\lipsum[1-2]

\subsection{Marine heatwaves}

\lipsum[1-2]

\begin{figuremulticol}
	\centering
	\includegraphics[width=\linewidth]{../figs/01_part-1/fig-5.png}
	\captionof{figure}{El Nino Southern Oscillation Index (SOI) from 1980 to 2023. Negative values of the index indicate El Nino events whereas positive values indicate La Nina events. The colored area are 6-months moving average, and the grey bar indicates true values.}
	\label{part1:chap2:fig-5}
\end{figuremulticol}

\lipsum[1-2]

\section{Cyclones}

\lipsum[1-2]

\begin{figuremulticol}
	\centering
	\includegraphics[width=\linewidth]{../figs/01_part-1/fig-8.png}
	\captionof{figure}{Comparison of the number of cyclones that passed within 100 km of a coral reef between countries and territories of the Pacific, from 1980 to 2023.}
	\label{part1:chap2:fig-7}
\end{figuremulticol}

\lipsum[1-2]

\end{multicols}

\begin{figure}[h!]
	\centering
	\includegraphics[width=\linewidth]{../figs/01_part-1/fig-7.png}
	\caption{Trajectories of cyclones that passed within 100 km of a coral reef of the Pacific, between 1980 and 2023. Polygons represents economic exclusive zones.}
	\label{part1:chap2:fig-6}
\end{figure}

\newpage

\begin{table}[h!]
	\caption{Human population in countries and territories}
	\input{../figs/01_part-1/table-3.tex}
	\label{part1:chap1:tbl-3}
\end{table}

\newpage

\begin{multicols}{2}
	
\section{Ocean acidification}

\lipsum[1-2]

\section{Crown-of-Thorns Starfish}

\lipsum[1]

\begin{figuremulticol}
	\centering
	\includegraphics[width=\linewidth]{../figs/01_part-1/fig-9.jpg}
	\captionof{figure}{Crown-of-Thorns Starfish eating hard corals on a South Tuamotu atoll, French Polynesia (2021-05-22). Credit: Yannick Chancerelle.}
	\label{part1:chap2:fig-8}
\end{figuremulticol}

\lipsum[1]

\section{Fishing}

\lipsum[1-2]

\section{Nutrients and sediments}

\lipsum[1-2]

\section{Pollution}

\lipsum[1-2]

\section{Non-native species}

\lipsum[1-2]

\section{Diseases}

\lipsum[1-2]

\end{multicols}

\newpage

\chapter{Temporal trends in benthic cover}

\setcounter{chaptercounter}{3}

\begin{multicols}{2}

\section{Spatio-temporal distribution of monitoring}

\lipsum[1-3]

\begin{figuremulticol}
	\centering
	\includegraphics[width=\linewidth]{../figs/01_part-1/fig-11.png}
	\captionof{figure}{Percentage of total number of surveys conducted per year.}
	\label{part1:chap2:fig-11}
\end{figuremulticol}

\lipsum[2]

\begin{figuremulticol}
	\centering
	\includegraphics[width=\linewidth]{../figs/01_part-1/fig-12.png}
	\captionof{figure}{Percentage of surveys conducted per depth in the GCRMN Pacific region.}
	\label{part1:chap2:fig-12}
\end{figuremulticol}

\lipsum[1]

\end{multicols}

\begin{figure}[h!]
	\centering
	\includegraphics[width=\linewidth]{../figs/01_part-1/fig-10.png}
	\caption{Spatio-temporal distribution of benthic cover monitoring sites across the Pacific region. Sites that were monitored for the longest period of time are represented on top of the sites monitored for fewer years. The grey polygon represents the overlapping claimed area of Matthew and Hunter Islands, by France and Vanuatu.}
	\label{part1:chap1:fig-9}
\end{figure}

\newpage

\begin{table}[h!]
	\caption{Comparison of monitoring descriptors between the territories of the GCRMN Pacific region}
	\input{../figs/01_part-1/table-4.tex}
	\label{part1:chap2:tbl-4}
\end{table}

\newpage

\begin{multicols}{2}
	
\section{Trends in major benthic categories}

\lipsum[1-2]

\begin{figuremulticol}
	\centering
	\includegraphics[width=\linewidth]{../figs/01_part-1/fig-13.png}
	\captionof{figure}{Estimated temporal trends for hard coral (A), coralline algae (B), macroalgae (C), and turf algae (D) cover. The bold line represents the estimated mean cover, lighter and darker ribbons represent 95\% and 80\% confidence intervals, respectively. Note the scale on y-axis are different between subplots.}
	\label{part1:chap2:fig-13}
\end{figuremulticol}

\lipsum[1-2]

\newpage

\section{Trends in major hard coral families}

\lipsum[1-2]

\end{multicols}

\begin{figure}[h!]
	\centering
	\includegraphics[width=\linewidth]{../figs/01_part-1/fig-14.png}
	\caption{Spatio-temporal distribution of benthic cover monitoring sites across the Pacific region. Sites that were monitored for the longest period of time are represented on top of the sites monitored for fewer years. The grey polygon represents the overlapping claimed area of Matthew and Hunter Islands, by France and Vanuatu.}
	\label{part1:chap1:fig-13}
\end{figure}

\lipsum[1-2]

\newpage

\chapter{Recommandations}

\setcounter{chaptercounter}{4}

\begin{multicols}{2}

\section{Reduction in greenhouse gas emissions}

\lipsum[1-2]

\section{Reduction of local threats}

\lipsum[1-2]

\section{Implementation of marine protected areas}

\lipsum[1-2]

\section{Reef rehabilitation}

\lipsum[1-2]

\section{Maintenance and development of monitoring}

\lipsum[1-2]

\end{multicols}

%%%%%%%%%%%%%%%%%%%%%%%%%%%%%%%%%%%%%%%%%%%%%%%%%%%%%%%%%%%%%%%%%%%%%%%%%%%%%%%%%%%%%%%%%%%%%%%%%%%%%%%%%%%%%%%%%%%%%%%%%%%%%%%%%%%%%%
%%%%% PART 2 %%%%%%%%%%%%%%%%%%%%%%%%%%%%%%%%%%%%%%%%%%%%%%%%%%%%%%%%%%%%%%%%%%%%%%%%%%%%%%%%%%%%%%%%%%%%%%%%%%%%%%%%%%%%%%%%%%%%%%%%%
%%%%%%%%%%%%%%%%%%%%%%%%%%%%%%%%%%%%%%%%%%%%%%%%%%%%%%%%%%%%%%%%%%%%%%%%%%%%%%%%%%%%%%%%%%%%%%%%%%%%%%%%%%%%%%%%%%%%%%%%%%%%%%%%%%%%%%

\newpage

\fancypagestyle{plain}{%
	\fancyhf{}%
	\fancyfoot[RO,LE]{}%
	\renewcommand{\headrulewidth}{0pt}%
}

\pagestyle{plain}

\titleformat{\part}
{\fontsize{40pt}{40pt}\selectfont}
{}
{0pt}
{\begin{tikzpicture}[remember picture,overlay]
		\node[opacity=1,inner sep=0pt] at (current page.center){\includegraphics[width=\paperwidth,height=\paperheight]{../figs/00_misc/part.png}};
		\node[anchor=west, fill,text centered, fill=firstcolor, opacity=1, inner xsep=0.65cm, inner ysep=0.65cm] at ($(current page.north west)+(2.5cm,-5cm)$) {{\fontsize{50pt}{40pt}\selectfont \textcolor{white}{\textit{2}}}};
		\node[anchor=west] at ($(current page.north west)+(2.15cm,-3.15cm)$) {{\fontsize{26pt}{40pt}\selectfont \textcolor{white}{PART}}};
		\fill[anchor=west, secondcolor, opacity=0.5] ($(current page.north west)+(-1cm,-6.5cm)$) rectangle ($(current page.north west)+(22cm,-10.5cm)$);
		\node[yshift=-3cm] at (current page.north west)
		{\begin{tikzpicture}[remember picture, overlay]
				(\paperwidth,10cm);
				\node[anchor=north west,xshift=2.5cm,yshift=-4cm,rectangle,inner sep=8pt, text width=15cm]
				{\color{firstcolor}\raggedleft#1};
			\end{tikzpicture}
		};
	\end{tikzpicture}
}
\titlespacing*{\part}{0pt}{50pt}{-10pt} % The last argument control space between chapter title and text

\part[Part 2. Syntheses for countries and territories]{Syntheses for countries and territories}

\newpage

\thispagestyle{empty}

\begin{tikzpicture}[remember picture,overlay]
	\fill[anchor=west, secondcolor] ($(current page.north west)$) rectangle ($(current page.south east)$);
\end{tikzpicture}

\newpage

\fancypagestyle{plain}{%
	\fancyhf{}%
	\fancyfoot[LE,RO]{\sffamily\footnotesize \textcolor{firstcolor}{\thepage}}
	\fancyfoot[LO]{\sffamily\footnotesize \textcolor{thirdcolor}{Status and Trends of Coral Reefs of the Pacific}}
	\fancyfoot[RE]{\sffamily\footnotesize \textcolor{thirdcolor}{\leftmark}}
}

\pagestyle{plain}

%%%%%%%%%%%%%%%%%%%%%%%%%%%%%%%%%%%%%%%%%%%%%%%%%%%%%%%%%%%%%%%%%%%%%%%%%%%%%%%%%%%%%%%%%%%%%%%%%%%%%%%%%%%%%

%%%%% REDEFINE %%%%%%%

% Redefine TOC appearance for chapters %%

\titlecontents{chapter}
[1.5em]
{\selectfont\color{black}}
{\contentslabel[\thecontentslabel]{1em}}
{}
{\titlerule*[0.75pc]{.}\contentspage} % dots
[\addvspace{1ex}]

% Redefine title appearance for chapters %%

\titleformat{\chapter}
{\fontsize{25pt}{40pt}\selectfont}
{}
{0pt}
{\begin{tikzpicture}[remember picture,overlay]
		\node[yshift=-2.5cm] at (current page.north west)
		{\begin{tikzpicture}[remember picture, overlay]
				(\paperwidth,3cm);
				\fill[fill=secondcolor,color=secondcolor] ($(current page.north west)+(2.51cm,-2.75cm)$) rectangle ($(current page.north west)+(18.5cm,-4cm)$);
				\node[anchor=north west,fill=firstcolor, minimum width=16cm, inner sep=10pt, align=left] at ($(current page.north west)+(2.5cm,-2.5cm)$)
				{\color{white}\uppercase{\filright#1}};
				%\fill[fill=white,color=white] ($(current page.north west)+(16.75cm,-6cm)$) circle (1.5cm);
				\node[opacity=1,inner sep=0pt, scale=10.5] at ($(current page.north west)+(16.75cm,-5.75cm)$){\includegraphics[width=\paperwidth]{../figs/02_part-2/fig-1/\territoryname.png}};
			\end{tikzpicture}
		};
	\end{tikzpicture}
}
\titlespacing*{\chapter}{0pt}{20pt}{-30pt}

\setcounter{chaptercounter}{1}
\renewcommand{\thefigure}{\arabic{part}.\arabic{chaptercounter}.\arabic{figure}}
\renewcommand{\thetable}{\arabic{part}.\arabic{chaptercounter}.\arabic{table}}

%%%%%%%%%%%%%%%%%%%%%

% AMERICAN SAMOA
%%%%%%%%%%%%%%%%%%%%%%%%%%%%%%%%%%%%%%%%%%%%%%%%%%%%%%%%%%%%%%%%%%%%%%%%%%%%%%%%%%%%%%%%%%%%%%%%%%%%%%%%%%%%%

\def \territoryname {american-samoa}

\chapter{American Samoa}

\vspace{0.2cm}

\colorbox{secondcolor}{\small \textbf{Authors}} \input{../figs/02_part-2/tbl-2/authors.tex}

\vspace{0.25cm}

\par\noindent\rule{12.75cm}{0.4pt}

\input{../figs/02_part-2/tbl-3/affiliations.tex}

\vspace{0.75cm}

\begin{multicols}{2}
	
	\section*{Introduction}
	
	\input{../figs/02_part-2/tbl-1/\territoryname.tex}
	
	\vspace{0.5cm}
	
	\lipsum[1-4]
	
	\lipsum[1]
	
	\begin{figuremulticol}
		\centering
		\includegraphics[width=\linewidth]{../figs/02_part-2/fig-2/\territoryname.png}
		\captionof{figure}{\titleFigureA{American Samoa}}
		\label{part2:chap1:fig-2}
	\end{figuremulticol}
	
	\lipsum[1-3]
	
\end{multicols}

\newpage

% FRENCH POLYNESIA
%%%%%%%%%%%%%%%%%%%%%%%%%%%%%%%%%%%%%%%%%%%%%%%%%%%%%%%%%%%%%%%%%%%%%%%%%%%%%%%%%%%%%%%%%%%%%%%%%%%%%%%%%%%%%

\def \territoryname {french-polynesia}

\chapter{French Polynesia}

\addtocounter{chaptercounter}{1}

\vspace{0.2cm}

\colorbox{secondcolor}{\small \textbf{Authors}} \input{../figs/02_part-2/tbl-2/authors.tex}

\vspace{0.25cm}

\par\noindent\rule{12.75cm}{0.4pt}

\input{../figs/02_part-2/tbl-3/affiliations.tex}

\vspace{0.75cm}

\begin{multicols}{2}
	
	\section*{Introduction}
			
	\input{../figs/02_part-2/tbl-1/\territoryname.tex}

	\vspace{0.5cm}
	
	\lipsum[1-4]
	
	\section*{Disturbance regime}
	
	\subsection*{Thermal regime}
	
	\begin{figuremulticol}
		\centering
		\includegraphics[width=\linewidth]{../figs/02_part-2/fig-2/\territoryname.png}
		\captionof{figure}{\titleFigureA{French Polynesia}}
		\label{part2:chap2:fig-1}
	\end{figuremulticol}
	
	\lipsum[1-4]
	
	 Integer tempus convallis augue. Etiam facilisis. Nunc elementum fermentum wisi. Aenean placerat. Ut imperdiet, enim sed gravida sollicitudin, felis odio placerat quam, ac pulvinar elit purus eget enim. Nunc vitae tortor. Proin tempus nibhsit amet nisl. Vivamus quis tortor vitae risus porta vehicula.  Proin tempus nibhsit amet nisl. Vivamus quis tortor vitae risus porta vehicula.
	
\lipsum[2]
	
\end{multicols}

\newpage

% NEW CALEDONIA
%%%%%%%%%%%%%%%%%%%%%%%%%%%%%%%%%%%%%%%%%%%%%%%%%%%%%%%%%%%%%%%%%%%%%%%%%%%%%%%%%%%%%%%%%%%%%%%%%%%%%%%%%%%%%

\def \territoryname {new-caledonia}

\chapter{New Caledonia}

\addtocounter{chaptercounter}{1}

\vspace{0.2cm}

\colorbox{secondcolor}{\small \textbf{Authors}} \input{../figs/02_part-2/tbl-2/authors.tex}

\vspace{0.25cm}

\par\noindent\rule{12.75cm}{0.4pt}

\input{../figs/02_part-2/tbl-3/affiliations.tex}

\vspace{0.75cm}

\begin{multicols}{2}
	
	\section*{Introduction}
		
	\input{../figs/02_part-2/tbl-1/\territoryname.tex}
	
	\vspace{0.5cm}
		
	\lipsum[1-5]
	
\end{multicols}

\newpage

% CASE STUDY 1
%%%%%%%%%%%%%%%%%%%%%%%%%%%%%%%%%%%%%%%%%%%%%%%%%%%%%%%%%%%%%%%%%%%%%%%%%%%%%%%%%%%%%%%%%%%%%%%%%%%%%%%%%%%%%

\begin{tikzpicture}[remember picture,overlay]
	\fill[anchor=west, secondcolor!30!white] ($(current page.north west)$) rectangle ($(current page.south east)$);
	\node[anchor=north west, fill=firstcolor, inner sep=8pt] at ($(current page.north west)+(2.5cm,-2.5cm)$) {{\fontsize{20pt}{40pt}\selectfont \color{white}{\textbf{CASE STUDY 1}}}};
	\node[] at ($(current page.north east)+(-6cm,-5cm)$){\includegraphics[width=8cm]{../figs/02_part-2/case-studies/new-caledonia_1.png}};
\end{tikzpicture}

\vspace*{1.5cm}

{\LARGE\selectfont \textcolor{firstcolor}{Effects of the implementation\\ of a marine protected area on\\ commercial reef fish species biomass\\ and richness in Ouano, New Caledonia}}

\vspace*{1cm}

\textbf{Authors}

Laurent Wantiez\textsuperscript{\textcolor{firstcolor}{1}}

\par\noindent\rule{14cm}{0.4pt}

{\footnotesize 
	
	\textsuperscript{\textcolor{firstcolor}{1}}University of New Caledonia, France
	
}

\vspace*{1cm}

Ouano Marine Protected Area (3,499 ha) was created in 1989 but the MPA is implemented and enforced since 2007. The MPA extends from the coastline to the outer slope. The objective is the conservation of the lagoon seascape (mangrove, soft-bottoms, coral reefs) in the context of sustainable development. Ouano MPA is also included in the Lagoons of New Caledonia: Reef Diversity and Associated Ecosystems, a UNESCO world heritage site since 2008. The MPAS effects were assessed for coral reef fish, macroinvertebrates and coral habitat, using a Before-After Control-Impact Paired Series (BACIPS) design. The sampling network includes two unprotected areas, located North and South of the MPA. Three surveys were conducted before enforcement (2004-2006), then 7 surveys after (2007-2020), all at the same season. Each survey consisted of 30 stations sampled on coral reefs (fringing reef, lagoon reef and inner barrier reef). Eighteen are located within the MPA and 12 outside on unprotected reefs. At each station, the reef flat, the reef crest and the reef slope were sampled.

The first significant MPAs effects were detected in 2009, three years after enforcement. Then they increased until 2014, and they were still efficient during the last survey in 2020 despite a lowering for several indicators. The commercial fish species increased in the MPA until 2014 while it remained stable outside (\autoref{fig:casestudy:newcaledonia}). The MPA effects are even more significant for the biomass of commercial species, with a constant increase since enforcement, compared to the stability recorded outside the MPA (\autoref{fig:casestudy:newcaledonia}). The main families benefiting from protection are the groupers and the acanthurids. The coral trout illustrates these positive effects. The species was observed only once in the unprotected zone and only twice in the MPA before enforcement. Now, the species is censused In the MPAs every survey since enforcement, with maximum densities and biomasses since 2014. Similar patterns are described for the humphead wrasse and the bluespine unicornfish. These three fish species are among the most targeted coral reef fish in the country. Macroinvertebrates also benefited from MPA effects, such as the commercial top shell, the giant clams and the lobsters. No MPA effect was detected for the coral habitat as the anthropogenic pressure was similar inside and outside the MPA.

\newpage

\begin{tikzpicture}[remember picture,overlay]
	\fill[anchor=west, secondcolor!30!white] ($(current page.north west)$) rectangle ($(current page.south east)$);
\end{tikzpicture}

\begin{figure}[h!]
	\centering
	\includegraphics[width=\linewidth]{../figs/02_part-2/case-studies/new-caledonia_2.png}
	\caption[Comparison of the temporal trends of biomass (A) and species richness (B) of commercial coral reef fish species within and outside the Marine Protected Area (MPA) of Ouano, in New Caledonia.]{Comparison of the temporal trends of biomass (A) and species richness (B) of commercial coral reef fish species within and outside the Marine Protected Area (MPA) of Ouano, in New Caledonia. Points represent the mean between stations within (\textit{n} = 18) and outside (\textit{n} = 12) the MPA, and error bars represent the mean $\pm$ standard error. The lines between the points are only intended to show the temporal trends and do not rely on data.}
	\label{fig:casestudy:newcaledonia}
\end{figure}

However, the level of these positive effects is directly related to the enforcement strategy. A change of the management of the MPA occurred in 2014. Before, the ranger?s office in charge of the MPA enforcement was located on top of a hill with a direct view on all the MPA. Similarly, any person in the MPA can see the office. In 2014, the office was moved to the village of La Foa. The presence of the ranges on the water increased but they had no view of the MPA from their office anymore. At the same time, we observed a stabilization or a decrease of the MPA effects on several indicators between 2014 and 2020, such as illustrated by the commercial fish species richness (\autoref{fig:casestudy:newcaledonia}). We hypotheses that the presence of the ranger's office in view of the MPA had a significant dissuasive impact on potential poachers. This highlights the importance of the perception of control strategy by the users of the area.

\newpage

% KIRIBATI
%%%%%%%%%%%%%%%%%%%%%%%%%%%%%%%%%%%%%%%%%%%%%%%%%%%%%%%%%%%%%%%%%%%%%%%%%%%%%%%%%%%%%%%%%%%%%%%%%%%%%%%%%%%%%

\def \territoryname {kiribati}

\chapter{Kiribati}

\addtocounter{chaptercounter}{1}

\vspace{0.2cm}

\colorbox{secondcolor}{\small \textbf{Authors}} \input{../figs/02_part-2/tbl-2/authors.tex}

\vspace{0.25cm}

\par\noindent\rule{12.75cm}{0.4pt}

\input{../figs/02_part-2/tbl-3/affiliations.tex}

\vspace{0.75cm}

\section*{Introduction}

\input{../figs/02_part-2/tbl-1/\territoryname.tex}
	
\vspace{0.5cm}
		
\begin{multicols}{2}
	
	\lipsum[2]
	
\end{multicols}

\newpage

% PACIFIC REMOTE ISLANDS AREA
%%%%%%%%%%%%%%%%%%%%%%%%%%%%%%%%%%%%%%%%%%%%%%%%%%%%%%%%%%%%%%%%%%%%%%%%%%%%%%%%%%%%%%%%%%%%%%%%%%%%%%%%%%%%%

\def \territoryname {pria}

\chapter{Pacific Remote Islands Area}

\addtocounter{chaptercounter}{1}

\vspace{0.2cm}

\colorbox{secondcolor}{\small \textbf{Authors}} \input{../figs/02_part-2/tbl-2/authors.tex}

\vspace{0.25cm}

\par\noindent\rule{12.75cm}{0.4pt}

\input{../figs/02_part-2/tbl-3/affiliations.tex}

\vspace{0.75cm}

\section*{Introduction}

\input{../figs/02_part-2/tbl-1/\territoryname.tex}

\vspace{0.5cm}

\begin{multicols}{2}
	
	\lipsum[2]
	
\end{multicols}

\newpage

% PAPUA NEW GUINEA
%%%%%%%%%%%%%%%%%%%%%%%%%%%%%%%%%%%%%%%%%%%%%%%%%%%%%%%%%%%%%%%%%%%%%%%%%%%%%%%%%%%%%%%%%%%%%%%%%%%%%%%%%%%%%

\def \territoryname {papua-new-guinea}

\chapter{Papua New Guinea}

\addtocounter{chaptercounter}{1}

\vspace{0.2cm}

\colorbox{secondcolor}{\small \textbf{Authors}} \input{../figs/02_part-2/tbl-2/authors.tex}

\vspace{0.25cm}

\par\noindent\rule{12.75cm}{0.4pt}

\input{../figs/02_part-2/tbl-3/affiliations.tex}

\vspace{0.75cm}

\begin{multicols}{2}
	
	\section*{Introduction}
	
	\input{../figs/02_part-2/tbl-1/\territoryname.tex}
	
	\vspace{0.5cm}
	
	\lipsum[1-3] \cite{lamy2016three}.
	
	\section*{Disturbance regime}
	
	\subsection*{Thermal regime}
	
	\lipsum[1-3]
	
	\subsection*{Cyclones}
	
	\lipsum[1-3]
		
	\subsection*{Other disturbances}
	
	\lipsum[1-3]
	
	\section*{Benthic cover}
		
	\subsection*{Monitoring}
	
	\lipsum[1-3]
			
	\subsection*{Temporal trends}
	
	\lipsum[1]
	
	\begin{figuremulticol}
		\centering
		\includegraphics[width=\linewidth]{../figs/02_part-2/fig-7/fiji.png}
		\captionof{figure}{testststs}
		\label{part2:chap8:fig-2}
	\end{figuremulticol}
		
\end{multicols} 

\newpage

% MATERIALS AND METHODS
%%%%%%%%%%%%%%%%%%%%%%%%%%%%%%%%%%%%%%%%%%%%%%%%%%%%%%%%%%%%%%%%%%%%%%%%%%%%%%%%%%%%%%%%%%%%%%%%%%%%%%%%%%%%%

%%%%% REDEFINE %%%%%%%

% Redefine TOC appearance for chapters %%

\titlecontents{chapter}
	[0em]
	{\selectfont\color{black}}
	{\contentslabel[\thecontentslabel]{1em}}
	{}
	{\titlerule*[0.75pc]{.}\contentspage} % dots
	[\addvspace{1ex}]

% Redefine title appearance for chapters %%

\titleformat{\chapter}
{\fontsize{20pt}{40pt}\selectfont}
{}
{0pt}
{\begin{tikzpicture}[remember picture,overlay]
		\node[yshift=-3cm] at (current page.north west)
		{\begin{tikzpicture}[remember picture, overlay]
				(\paperwidth,3cm);
				\node[anchor=north west,text centered,xshift=2.5cm,yshift=-0.4cm,rectangle,inner sep=8pt, 
				fill=firstcolor,draw=white, ultra thick]
				{\color{white}\uppercase{\filright#1}};
			\end{tikzpicture}
		};
	\end{tikzpicture}
}
\titlespacing*{\chapter}{0pt}{50pt}{-10pt} % The last argument control space between chapter title and text

\setcounter{part}{3}
\renewcommand{\thefigure}{\arabic{part}.\arabic{figure}}
\renewcommand{\thetable}{\arabic{part}.\arabic{table}}

%%%%%%%%%%%%%%%%%%%%%%%%%%%

\begin{tikzpicture}[remember picture,overlay]
	\node[opacity=1,inner sep=0pt, anchor=north west] at ($(current page.north west)+(0cm,0cm)$){\includegraphics[width=\paperwidth]{../figs/00_misc/chapter.png}};
	\node[anchor=west] at ($(current page.north west)+(2.5cm,-0.75cm)$) {\textcolor{white}{{\scriptsize Hannes Klostermann. Coral reefs in Moorea, French Polynesia}}};
\end{tikzpicture}

\bookmarksetup{startatroot} % To leave the previous part section (for bookmarks in the PDF)

\chapter{Materials and Methods}

\begin{multicols}{2}
	
	\section{Background maps}
	
	\lipsum[1-4]
	
	\begin{equation}
		\label{eq:1}
		\mathrm{R^{2} = 1 - \frac{\left( y_{i} - \hat{y}_{i} \right)^{2}}{\left( y_{i} - \bar{y} \right)^{2}}}
	\end{equation}
	
	where $\bar{y}$ is the average of predictions, $\hat{y}_{i}$ is the predicted value of cover for observation $i$ \eqref{eq:1}.
	
	\lipsum[1]
	
	\section{Geographic indicators}
	
	\subsection{Maritime area}
	
	\section{Cyclones}
	
	\section{Thermal regime}
	
	\section{Benthic cover}
	
\end{multicols}

Using the Allen Coral Atlas data \citep{lyons2024new}.

\begin{equation}
	\mathnormal{RMSE = \sqrt{\sum_{i=1}^{n} \frac{\left( \hat{y}_{i} - y_{i} \right)^{2}}{n}}}
\end{equation}

\newpage

\begin{figure}[h!]
	\centering
	\vspace*{0.5cm}
	\begin{tikzpicture}
		
		% Define styles
		
		\tikzstyle{arrow1}=[thick,color=black,-latex]
		\tikzstyle{circletext}=[circle,radius=0.15,fill=firstcolor,font=\bfseries,text=white]
		\tikzstyle{circletextlabel}=[anchor=west,xshift=0.75cm,text=firstcolor]
		\tikzstyle{circletextsublabel}=[text width=5cm,text=darkgray,font=\footnotesize,anchor=west,xshift=0.75cm,yshift=-1cm]
		\tikzstyle{datasource}=[rectangle,inner sep=3pt, fill=secondcolor, text=black, font=\footnotesize, rounded corners]
		\tikzset{
			pics/table1/.style args={#1,#2,#3}{
				code={
					\fill[rectangle,fill=#3, draw=white, thick] (#1,#2) rectangle (#1+0.3,#2-0.3);
					\fill[rectangle,fill=#3, draw=white, thick] (#1+0.3,#2) rectangle (#1+0.6,#2-0.3);
					\fill[rectangle,fill=#3, draw=white, thick] (#1+0.6,#2) rectangle (#1+0.9,#2-0.3);
					\fill[rectangle,fill=#3, draw=white, thick] (#1+0.9,#2) rectangle (#1+1.2,#2-0.3);
					\fill[rectangle,fill=#3, draw=white, thick] (#1+1.2,#2) rectangle (#1+1.5,#2-0.3);

					\fill[rectangle,fill=lightgray, draw=white, thick] (#1,#2-0.3) rectangle (#1+0.3,#2-0.6);
					\fill[rectangle,fill=lightgray, draw=white, thick] (#1+0.3,#2-0.3) rectangle (#1+0.6,#2-0.6);
					\fill[rectangle,fill=lightgray, draw=white, thick] (#1+0.6,#2-0.3) rectangle (#1+0.9,#2-0.6);
					\fill[rectangle,fill=lightgray, draw=white, thick] (#1+0.9,#2-0.3) rectangle (#1+1.2,#2-0.6);
					\fill[rectangle,fill=lightgray, draw=white, thick] (#1+1.2,#2-0.3) rectangle (#1+1.5,#2-0.6);
					
					\fill[rectangle,fill=lightgray, draw=white, thick] (#1,#2-0.6) rectangle (#1+0.3,#2-0.9);
					\fill[rectangle,fill=lightgray, draw=white, thick] (#1+0.3,#2-0.6) rectangle (#1+0.6,#2-0.9);
					\fill[rectangle,fill=lightgray, draw=white, thick] (#1+0.6,#2-0.6) rectangle (#1+0.9,#2-0.9);
					\fill[rectangle,fill=lightgray, draw=white, thick] (#1+0.9,#2-0.6) rectangle (#1+1.2,#2-0.9);
					\fill[rectangle,fill=lightgray, draw=white, thick] (#1+1.2,#2-0.6) rectangle (#1+1.5,#2-0.9);

				}
			}
		}
		\tikzset{
			pics/table2/.style args={#1,#2,#3}{
				code={
					\fill[rectangle,fill=#3, draw=white, thick] (#1,#2) rectangle (#1+0.3,#2-0.3);
					
					\fill[rectangle,fill=lightgray, draw=white, thick] (#1,#2-0.3) rectangle (#1+0.3,#2-0.6);
					
					\fill[rectangle,fill=lightgray, draw=white, thick] (#1,#2-0.6) rectangle (#1+0.3,#2-0.9);
					
				}
			}
		}
		
		% STEP 1 - Standardization
			
		\node[circletext] at (0,0) {1};
		\node[circletextlabel] at (0,0) {Standardization};
		\node[circletextsublabel] at (0,0) {Standardize all individual datasets to the same format};
		
		\draw pic{table1={-8.5, 0, secondcolor!60!firstcolor}};
		\draw pic{table1={-6, 0, secondcolor!40!firstcolor}};
		\draw pic{table1={-3.5, 0, secondcolor!80!firstcolor}};
		
		\node[circletextsublabel, centered] at (-6.68,1.4) {dataset 1};
		\node[circletextsublabel, centered] at (-4.18,1.4) {dataset 2};
		\node[circletextsublabel, centered] at (-1.68,1.4) {dataset \textit{i}};
		
		\draw[arrow1] (-7.75,-1) -> (-7.75,-1.9);
		\draw[arrow1] (-5.25,-1) -> (-5.25,-1.9);
		\draw[arrow1] (-2.75,-1) -> (-2.75,-1.9);
		
		% STEP 2 - Grouping
				
		\node[circletext] at (0,-3) {2};
		\node[circletextlabel] at (0, -3) {Grouping};
		\node[circletextsublabel] at (0,-3) {Bind all standardized individual datasets together};
				
		\draw pic{table1={-8.5, -2, firstcolor}};
		\draw pic{table1={-6, -2, firstcolor}};
		\draw pic{table1={-3.5, -2, firstcolor}};
				
		\draw[arrow1] (-5.25,-3) -> (-5.25,-3.9);
				
		% STEP 3 - Taxonomic recatgorization
						
		\node[circletext] at (0,-6) {3};
		\node[circletextlabel] at (0, -6) {Taxonomic recategorization};
		\node[circletextsublabel] at (0,-6) {Correct, recategorize, and find upper taxonomic levels};
				
		\draw[thick,color=black] (-7.75, -3) -> (-7.75, -3.45) -> (-5.25, -3.45) -> (-2.75, -3.45) -> (-2.75, -3);
				
		\draw pic{table1={-6, -4, firstcolor}};
				
		\draw[arrow1] (-4.35,-4.45) -> (-2.75,-4.45) -> (-2.75,-7.9);
				
		\draw[arrow1] (-6.15,-4.45) -> (-8.35,-4.45) -> (-8.35,-5.9);
		
		\draw[thick,color=black] (-6.5,-7.35) -> (-8.35,-7.35);
		
		\node[datasource] at (-6.5,-7.35) {WoRMS};
		
		\node[circletextsublabel, text width=2.4cm] at (-8.5,-5.25) {Raw benthic categories};
				
		\draw pic{table2={-8.5, -6, secondcolor!40!firstcolor}};
				
		\draw[arrow1] (-8.35,-7) -> (-8.35,-7.9);
				
		\draw pic{table1={-8.5, -8, secondcolor!80!firstcolor}};
		
		\draw pic{table2={-8.5, -8, secondcolor!40!firstcolor}};
		
		\node[circletextsublabel, align=left, text width=2cm] at (-9.25,-8.3) {CSV file};
		
		\node[circletextsublabel, align=center, text width=2cm] at (-7.3,-7.9) {Join};
						
		\draw pic{table1={-3.5, -8, firstcolor}};
				
		\draw[arrow1] (-2.75,-9) -> (-2.75,-12.9);
		
		\draw[arrow1] (-6.85,-8.45) -> (-3.7,-8.45);
				
		% STEP 4 - Spatial attribution
						
		\node[circletext] at (0,-10.5) {4};
		\node[circletextlabel] at (0, -10.5) {Spatial attribution};
		\node[circletextsublabel] at (0,-10.5) {Assign GCRMN region, country, and territory to each site};
				
		\draw[thick,color=black] (-5,-10.3) -> (-2.75,-10.3);
		\node[datasource] at (-5,-10.3) {GCRMN region};

		\draw[thick,color=black] (-5,-11.6) -> (-2.75,-11.6);
		\node[datasource] at (-5,-11.6) {EEZ};
				
		\draw pic{table1={-3.5, -13, firstcolor}};
				
		\draw[arrow1] (-2.75,-14) -> (-2.75,-15.9);
		
		\draw[arrow1] (-2.75,-14.85) -> (-7.75,-14.85) -> (-7.75, -15.9);
		
		\draw[arrow1] (-2.75,-14.85) -> (-7.75,-14.85) -> (-7.75, -13.7);
				
		% STEP 5 - Quality checks
						
		\node[circletext] at (0,-14.5) {5};
		\node[circletextlabel] at (0, -14.5) {Quality checks};
		\node[circletextsublabel] at (0,-14.5) {Control data quality and remove incorrect rows};
							
		\draw pic{table1={-8.5, -16, fuzzyred}};
									
		\draw pic{table1={-3.5, -16, firstcolor}};
		
		\node[circletextsublabel, align=center, text width=2cm] at (-4.65,-16.6) {Synthetic dataset};
		\node[circletextsublabel, align=center, text width=2cm] at (-9.65,-16.6) {Removed data};
		
		\fill[rectangle,fill=lightgray,draw=white,thick] (-8.1,-12.4) rectangle (-7.3,-13.4);
		\fill[rectangle,fill=lightgray,draw=white,thick] (-8.2,-12.5) rectangle (-7.4,-13.5);
		\draw[thick,color=fuzzyred] (-8,-12.7) -> (-7.6,-12.7);
		\draw[thick,color=white] (-7.9,-12.9) -> (-7.6,-12.9);
		\draw[thick,color=white] (-7.9,-13) -> (-7.6,-13);
		\draw[thick,color=white] (-7.9,-13.1) -> (-7.6,-13.1);
		\draw[thick,color=white] (-7.9,-13.2) -> (-7.6,-13.2);
		\draw[thick,color=white] (-7.9,-13.3) -> (-7.6,-13.3);
		
		\node[circletextsublabel, align=left, text width=2.5cm] at (-7.8,-12.1) {Quality checks reports};
				
	\end{tikzpicture}
	\vspace*{0.5cm}
	\caption[Illustration of the data integration workflow used for the creation of the gcrmndb benthos synthetic dataset]{Illustration of the data integration workflow used for the creation of the gcrmndb benthos synthetic dataset.}
	\label{metm:fig-1}
\end{figure}

\newpage

\begin{table}[]
	\caption[Description of variables included in the synthetic dataset]{Description of variables included in the synthetic dataset. Variable names (except category, subcategory, and condition) correspond to \href{https://dwc.tdwg.org/terms/}{DarwinCore terms}. Variables 2 to 11 are associated with spatial dimension, variables 12 to 15 are associated with temporal dimension, and variables 18 to 26 are associated with taxonomy.}
	\begin{center}
		\begin{tabular}{|C{0.75cm}|C{4cm}|C{2.5cm}|L{7cm}|}
			\hline
			\rowcolor{firstcolor}
			\textcolor{white}{Nb} & \textcolor{white}{Variable} & \textcolor{white}{Type} & \textcolor{white}{Description} \\ \hline
			\rowcolor{white}
			1              &          datasetID          &         Factor          & ID of the dataset                    \\ \hline
			\rowcolor{secondcolor}
			2            &       higherGeography       &         Factor          & GCRMN region                   \\ \hline
			\rowcolor{white}
			3              &          country          &         Factor          & Country                     \\ \hline
			\rowcolor{secondcolor}
			4            &       territory       &         Character          & Territory                  \\ \hline
			\rowcolor{white}
			5              &          locality          &         Character          & Site name
			                     \\ \hline
			\rowcolor{secondcolor}
			6            &       habitat       &         Factor          & Habitat                   \\ \hline
			\rowcolor{white}
			7              &          parentEventID          &         Integer          & Transect ID
			                     \\ \hline
			\rowcolor{secondcolor}
			8            &       eventID       &         Integer          & Quadrat ID
			                  \\ \hline
			\rowcolor{white}
			9              &          decimalLatitude          &         Numeric          & Latitude (decimal, EPSG:4326)
			                     \\ \hline
			\rowcolor{secondcolor}
			10            &       decimalLongitude       &         Numeric          & Longitude (decimal, EPSG:4326)
			                   \\ \hline
			\rowcolor{white}
			11             &          verbatimDepth          &         Numeric          & Depth (m)                     \\ \hline
			\rowcolor{secondcolor}
			12           &       year       &         Integer          & Four-digit year
			                   \\ \hline
			\rowcolor{white}
			13             &          month          &         Integer          & Integer month
			                     \\ \hline
			\rowcolor{secondcolor}
			14           &       day       &         Integer          & Integer day
			                   \\ \hline
			\rowcolor{white}
			15             &          eventDate          &         Date          & Date (YYYY-MM-DD, ISO 8601)
			                     \\ \hline
			\rowcolor{secondcolor}
			16           &       samplingProtocol       &         Character          &  Method used to acquire the measurement                  \\ \hline
			\rowcolor{white}
			17             &          recordedBy          &         Character          &  Person who acquired the measurement                    \\ \hline
			\rowcolor{secondcolor}
			18           &       category       &         Factor          & Benthic category
			                  \\ \hline
			\rowcolor{white}
			19             &          subcategory          &         Factor          & Benthic subcategory
			                     \\ \hline
				\rowcolor{secondcolor}
			20           &       condition       &         Character          &  Condition for hard corals                  \\ \hline
			\rowcolor{white}
			21             &          phylum          &         Character          & Phylum                     \\ \hline
				\rowcolor{secondcolor}
			22           &       class       &         Character          & Class                   \\ \hline
			\rowcolor{white}
			23             &          order          &         Character          & Order                     \\ \hline
				\rowcolor{secondcolor}
			24           &       family       &         Character          & Family                   \\ \hline
			\rowcolor{white}
			25             &          genus          &         Character          & Genus                     \\ \hline
				\rowcolor{secondcolor}
			26           &       scientificName       &         Character          & Species                   \\ \hline
			\rowcolor{white}
			27             &          measurementValue          &         Numeric          & Percentage cover
			                     \\ \hline
		\end{tabular}
	\end{center}
	\label{part3:tbl-1}
	\vspace*{0.5cm}
\end{table}

\newpage

\begin{table}[]
	\caption[List of quality checks used for the gcrmndb benthos synthetic dataset.]{List of quality checks used for the gcrmndb benthos synthetic dataset. Inspired by Vandepitte et al, 2015. EEZ = Economic Exclusive Zone.}
\begin{center}
	\begin{tabular}{|C{0.75cm}|C{4cm}|L{9.9cm}|}
		\hline
		\rowcolor{firstcolor}
		\textcolor{white}{Nb} & \textcolor{white}{Variable(s)} & \textcolor{white}{Question} \\
		\hline
		\rowcolor{white}
		1 & decimalLatitude \newline decimalLongitude & Are the latitude and longitude available?  \\ \hline
		\rowcolor{secondcolor}
		2 & decimalLatitude & Is the latitude within its possible boundaries \newline (\ie between -90 and 90)?  \\ \hline
		\rowcolor{white}
		3 & decimalLongitude & Is the longitude within its possible boundaries \newline (\ie between -180 and 180)?  \\ \hline
		\rowcolor{secondcolor}
		4 & decimalLatitude \newline decimalLongitude & Is the site within the coral reef distribution area \newline (100 km buffer)?  \\ \hline
		\rowcolor{white}
		5 & decimalLatitude \newline decimalLongitude & Is the site located within an EEZ (1 km buffer)? \\ \hline
		\rowcolor{secondcolor}
		6 & year & Is the year available?  \\ \hline
		\rowcolor{white}
		7 & measurementValue & Is the sum of the percentage cover of benthic categories within the sampling unit greater than 0 and lower than 100? \\ \hline
		\rowcolor{secondcolor}
		8 & measurementValue & Is the percentage cover of a given benthic category \newline (\ie a row) greater than 0 and lower than 100? \\ \hline
	\end{tabular}
\end{center}
	\label{part3:tbl-2}
	\vspace*{0.5cm}
\end{table}

\begin{multicols}{2}
	
	\section{Code availibility}
	
	\lipsum[1-2]
	
\end{multicols}

\newpage

\begin{multicols*}{2}
	
\section{Models}

\textbf{1. \textcolor{firstcolor}{human pop}}

\vspace{0.25cm}

\hrule

Category:

Data source:

Spatial resolution:

	


\end{multicols*}

\newpage

\begin{figure}
	\begin{center}
		\begin{tikzpicture}
			\tikzstyle{arrow1}=[thick,color=black,-latex]
			\tikzstyle{arrow2}=[thin,color=black,densely dashed]
			\tikzstyle{circletext}=[circle,radius=0.15,fill=firstcolor,font=\bfseries,text=white]
			\tikzstyle{circletextlabel}=[anchor=west,xshift=0.75cm,text=firstcolor]
			\tikzstyle{circletextsublabel}=[text width=5cm,text=darkgray,font=\footnotesize,anchor=west,xshift=0.75cm,yshift=-1cm]
			% Grid without points
			\tikzset{
				pics/layer/.style args={#1,#2,#3}{
					code={
						\fill[draw=black, thin, fill=fifthcolor, opacity=#3] (#1-2, #2) -- (#1,#2-1) -- (#1+2,#2) --  (#1,#2+1) -- (#1-2,#2);
						
					}
				}
			}
			\tikzset{
				pics/siteleft/.style args={#1,#2}{
					code={
						\fill[fill=firstcolor!60!white, draw=firstcolor] (#1,#2-0.5) circle[radius=2pt];
						\fill[fill=firstcolor!60!white, draw=firstcolor] (#1+0.5, #2+0.25) circle[radius=2pt];
						\fill[fill=firstcolor!60!white, draw=firstcolor] (#1-0.5, #2-0.25) circle[radius=2pt];
						
					}
				}
			}
			% Grid with points
			\tikzset{
				pics/layerpoints/.style args={#1,#2,#3}{
					code={
						\fill[draw=black, thin, fill=fifthcolor, opacity=#3] (#1-2, #2) -- (#1,#2-1) -- (#1+2,#2) --  (#1,#2+1) -- (#1-2,#2);
						
						\fill[fill=fuzzyred!60!white, draw=fuzzyred] (#1,#2) circle[radius=2pt];
						
						\fill[fill=fuzzyred!60!white, draw=fuzzyred] (#1,#2-0.5) circle[radius=2pt];
						\fill[fill=fuzzyred!60!white, draw=fuzzyred] (#1,#2+0.5) circle[radius=2pt];
						
						\fill[fill=fuzzyred!60!white, draw=fuzzyred] (#1-1,#2) circle[radius=2pt];
						\fill[fill=fuzzyred!60!white, draw=fuzzyred] (#1+1,#2) circle[radius=2pt];
						
						\fill[fill=fuzzyred!60!white, draw=fuzzyred] (#1-0.5, #2+0.25) circle[radius=2pt];
						\fill[fill=fuzzyred!60!white, draw=fuzzyred] (#1+0.5, #2+0.25) circle[radius=2pt];
						\fill[fill=fuzzyred!60!white, draw=fuzzyred] (#1-0.5, #2-0.25) circle[radius=2pt];
						\fill[fill=fuzzyred!60!white, draw=fuzzyred] (#1+0.5, #2-0.25) circle[radius=2pt];
						
					}
				}
			}
			\tikzset{
				pics/tablecol/.style args={#1,#2,#3}{
					code={
						\fill[rectangle,fill=#3, draw=white, thick] (#1,#2) rectangle (#1+0.3,#2-0.3);
						
						\fill[rectangle,fill=lightgray, draw=white, thick] (#1,#2-0.3) rectangle (#1+0.3,#2-0.6);
						
						\fill[rectangle,fill=lightgray, draw=white, thick] (#1,#2-0.6) rectangle (#1+0.3,#2-0.9);
						
					}
				}
			}
			
			% STEP 1 - Selection of benthic data
			
			\node[circletext] at (-2.5,9.5) {1};
			\node[circletextlabel] at (-2.5, 9.5) {Pre-processing of benthic data};
			\draw pic{layer={0,7.5,1}};
			\draw pic{siteleft={0, 7.5}};
			
			\draw pic{tablecol={1,5.5,firstcolor}};
			\node[circletextsublabel] at (0.155, 6.8) {$y$};
			\draw pic{tablecol={1.5,5.5,firstcolor!70!white}};
			\node[circletextsublabel] at (0.7, 6.8) {$pred.$};
			
			\node[circletextsublabel, text width=3cm] at (1.1, 9.35) {Site with observed data ($y$)};
			\draw [arrow1, color=darkgray] (1.8, 8.6) to [bend right=25] (0.55,7.9);
			
			\draw pic{tablecol={10.9,5.5,fuzzyred!70!white}};
			\node[circletextsublabel] at (10.05, 6.8) {$pred.$};
			
			% STEP 2 - Selection of benthic data
			
			\node[circletext] at (7.5,9.5) {2};
			\node[circletextlabel] at (7.5, 9.5) {Selection of sites to predict};
			\draw pic{layerpoints={10,7.5,1}};
			
			\node[circletextsublabel, text width=2cm, align=right] at (6, 7) {Site with coral reefs};
			\draw [arrow1, color=darkgray] (8.2, 6.55) to [bend left=20] (9.35, 7.25);
			
			% STEP 3 - Extraction of predictors
			
			% Left
			
			\draw[arrow1] (0, 6) -> (0,3.25);
			\draw[arrow1] (10, 6) -> (10,3.25);
			
			\node[circletext] at (2.5,3.5) {3};
			\node[circletextlabel] at (2.5, 3.5) {Extraction of predictors};
			
			\draw pic{layer={0,0,1}};
			\draw pic{siteleft={0, 0}};
			
			\draw pic{layer={0,1,0.9}};
			\draw pic{layer={0,1.25,0.9}};
			\draw pic{layer={0,1.5,0.9}};
			\draw pic{layer={0,1.75,0.9}};
			
			\draw pic{tablecol={1,-2,firstcolor}};
			\node[circletextsublabel] at (0.155, -0.7) {$y$};
			\draw pic{tablecol={1.5,-2,firstcolor!70!white}};
			\draw pic{tablecol={1.8,-2,firstcolor!70!white}};
			\draw pic{tablecol={2.1,-2,firstcolor!70!white}};
			\draw pic{tablecol={2.4,-2,firstcolor!70!white}};
			\draw pic{tablecol={2.7,-2,firstcolor!70!white}};
			\node[circletextsublabel] at (0.7, -0.7) {$pred.$};
			
			% Right
			
			\draw pic{layerpoints={10,0,1}};
			
			\draw pic{layer={10,1,0.9}};
			\draw pic{layer={10,1.25,0.9}};
			\draw pic{layer={10,1.5,0.9}};
			\draw pic{layer={10,1.75,0.9}};
			
			\node[circletextsublabel, centered] at (5.1, 2.35) {Predictor's data layers};
			\draw[black, thick] (2.3, 1.85) -> (2.4, 1.85) -> (2.4,0.85) -> (2.3,0.85);
			\draw[black, thick] (2.4, 1.35) -> (2.8, 1.35);
			\draw[black, thick] (7.7, 1.85) -> (7.6, 1.85) -> (7.6,0.85) -> (7.7,0.85);
			\draw[black, thick] (7.2, 1.35) -> (7.6, 1.35);
			
			\draw pic{tablecol={7.6,-2,fuzzyred!70!white}};
			\draw pic{tablecol={7.9,-2,fuzzyred!70!white}};
			\draw pic{tablecol={8.2,-2,fuzzyred!70!white}};
			\draw pic{tablecol={8.5,-2,fuzzyred!70!white}};
			\draw pic{tablecol={8.8,-2,fuzzyred!70!white}};
			\node[circletextsublabel] at (6.8, -0.7) {$pred.$};
			
			% STEP 4 - Machine learning model
			
			\draw[arrow1] (0, -1.5) -> (0,-3.25);
			
			\node[circletext] at (-2.5,-4) {4};
			\node[circletextlabel] at (-2.5, -4) {Machine learning model};
			
			\draw pic{tablecol={1,-5.5,firstcolor}};
			\node[circletextsublabel] at (0.155, -4.15) {$y$};
			\node[circletextsublabel] at (0.61, -5) {$\thicksim$};
			\draw pic{tablecol={2,-5.5,firstcolor!70!white}};
			\draw pic{tablecol={2.3,-5.5,firstcolor!70!white}};
			\draw pic{tablecol={2.6,-5.5,firstcolor!70!white}};
			\draw pic{tablecol={2.9,-5.5,firstcolor!70!white}};
			\draw pic{tablecol={3.2,-5.5,firstcolor!70!white}};
			\node[circletextsublabel] at (1.2, -4.15) {$pred.$};
			
			\draw [arrow1, color=darkgray] (2.3, -7.3) to [bend left=30] (1.6, -6.3);
			\node[circletextsublabel, text width=3.5cm] at (1.7, -6.75) {Relationship between $y$ and predictors' values};
			
			% STEP 5 - Model evaluation
			
			\draw[arrow1] (0, -4.6) -> (0,-7);
			
			\node[circletext] at (-2,-8) {5};
			\node[circletextlabel] at (-2, -8) {Model evaluation};
			
			\node[circletextsublabel] at (-2, -8.15) {Model performance};
			\node[circletextsublabel] at (-2, -8.65) {Model interpretation};
			
			% STEP 6 - Predictions
			
			\draw[black, thick] (3, -4) -> (10,-4);
			\draw[arrow1] (10, -1.5) -> (10,-7);
			
			\node[circletext] at (8.5,-8) {6};
			\node[circletextlabel] at (8.5, -8) {Predictions};
			
			\draw pic{tablecol={8.2,-9.5,fuzzyred!70!white}};
			\draw pic{tablecol={8.5,-9.5,fuzzyred!70!white}};
			\draw pic{tablecol={8.8,-9.5,fuzzyred!70!white}};
			\draw pic{tablecol={9.1,-9.5,fuzzyred!70!white}};
			\draw pic{tablecol={9.4,-9.5,fuzzyred!70!white}};
			\node[circletextsublabel] at (7.4, -8.2) {$pred.$};
			
			\draw[arrow1] (9.85, -9.95) -> (10.25,-9.95);
			
			\draw pic{tablecol={10.4,-9.5,fuzzyred}};
			\node[circletextsublabel] at (9.6, -8.2) {$\hat{y}$};
			
		\end{tikzpicture}
	\end{center}
	\vspace*{1cm}
	\caption[Illustration of the data integration workflow used for the creation of the gcrmndb benthos synthetic dataset]{Main steps of the machine learning modelling approach used to estimate the temporal trends of percentage cover of the four benthic categories (hard coral, macroalgae, turf algae, coralline algae). $y$ are the observed values of percentage cover (\textit{i.e.} ``measurementValue") for a given benthic category, $pred.$ is the set of predictors used to in the ML model, $\hat{y}$ are the predicted values of percentage cover for a given benthic category over coral reef sites.}
	\label{metm:fig-4}
\end{figure}

\newpage


\begin{table}[h!]
	\caption{Comparison of the number of observations used to train (column ``Training") and test (column ``Testing") the models used to estimate temporal trends for each of the four major benthic categories and the three major hard coral families. The total number of rows of the initial dataset is equal to the sum of values of the two columns for a given category.}
	\input{../figs/03_methods/table-3.tex}
	\label{part3:tbl-3}
\end{table}

\begin{table}[h!]
	\caption{Values of the tuned parameters of the model for each of the four major benthic categories and the three major hard coral families. ``Nb trees" is the number of trees, ``Min. obs." is the minimum observation per leaf.}
	\input{../figs/03_methods/table-4.tex}
	\label{part3:tbl-4}
\end{table}



\newpage
\clearpage

% AUTHORS' CONTRIBUTIONS
%%%%%%%%%%%%%%%%%%%%%%%%%%%%%%%%%%%%%%%%%%%%%%%%%%%%%%%%%%%%%%%%%%%%%%%%%%%%%%%%%%%%%%%%%%%%%%%%%%%%%%%%%%%%%

\setcounter{part}{4}
\renewcommand{\thefigure}{\arabic{part}.\arabic{figure}}
\renewcommand{\thetable}{\arabic{part}.\arabic{table}}

\begin{tikzpicture}[remember picture,overlay]
	\node[opacity=1,inner sep=0pt, anchor=north west] at ($(current page.north west)+(0cm,0cm)$){\includegraphics[width=\paperwidth]{../figs/00_misc/chapter.png}};
	\node[anchor=west] at ($(current page.north west)+(2.5cm,-0.75cm)$) {\textcolor{white}{{\scriptsize Hannes Klostermann. Coral reefs in Moorea, French Polynesia}}};
\end{tikzpicture}

\chapter{Author contributions}

\begin{multicols}{2}
	
	\lipsum[1]
	
\end{multicols}

\begin{table}[h]
	\caption[Authors' contribution to the production of the current report]{Authors' contribution to the production of the current report. ``Workshop" correspond to the participation to the workshop held in Auckland in November 2023 and dedicated to the production of this report. ``Mat. and methods" correspond to the writing and review of the materials and methods chapter. A contribution is denoted by \contrib\ for all columns, except columns associated to writing and review, where contributions are denoted by W and R, respectively.}
	\vspace{1cm}
		\resizebox{15cm}{!}{%
		\begin{tabular}{rlcccccccccccccccc}
			                                &          & \rot{Funding acquisition} & \rot{Supervision} & \rot{Conceptualization} & \rot{Facilitation} & \rot{Data acquisition} & \rot{Data integration} & \rot{Data analysis} & \rot{Workshop} & \rot{Writing and review} & \rot{\textcolor{firstcolor}{Exe. summary}} & \rot{\textcolor{firstcolor}{Synth. Pacific}} & \rot{\textcolor{firstcolor}{Synth. countries}} & \rot{\textcolor{firstcolor}{Case studies}} & \rot{\textcolor{firstcolor}{Mat. and methods}} & \rot{Layout} & \rot{Communication} \\ \hline
			  \rowcolor{secondcolor} Thomas & Dallison &                           &                   &                         &                    &                        &                        &                     &                &                          &                                            &                                              &                                                &                                     &                                                &              &      \contrib       \\ \hline
			         \rowcolor{white} Serge & Planes   &         \contrib          &     \contrib      &        \contrib         &                    &                        &                        &                     &                &                          &                                            &                                              &                                                &                                     &                                                &              &                     \\ \hline
			\rowcolor{secondcolor} Erica K. & Towle    &                           &                   &                         &      \contrib      &                        &                        &                     &                &                          &                                            &                                              &                                                &                                     &                    \review                     &              &                     \\ \hline
			        \rowcolor{white} J�r�my & Wicquart &                           &     \contrib      &        \contrib         &                    &                        &        \contrib        &      \contrib       &   \contrib             &                          &                                            &                                              &                                                &                                     &                    \writing                    &   \contrib   &                     \\ \hline
		\end{tabular}}
	\label{authors-contribution}
\end{table}

\newpage

% ANNEXES
%%%%%%%%%%%%%%%%%%%%%%%%%%%%%%%%%%%%%%%%%%%%%%%%%%%%%%%%%%%%%%%%%%%%%%%%%%%%%%%%%%%%%%%%%%%%%%%%%%%%%%%%%%%%%

\renewcommand{\thefigure}{S.\arabic{figure}.A}
\renewcommand{\thetable}{S.\arabic{table}}

\begin{tikzpicture}[remember picture,overlay]
	\node[opacity=1,inner sep=0pt, anchor=north west] at ($(current page.north west)+(0cm,0cm)$){\includegraphics[width=\paperwidth]{../figs/00_misc/chapter.png}};
	\node[anchor=west] at ($(current page.north west)+(2.5cm,-0.75cm)$) {\textcolor{white}{{\scriptsize Hannes Klostermann. Coral reefs in Moorea, French Polynesia}}};
\end{tikzpicture}

\chapter{Annexes}

\begin{figure}[h!]
	\centering
	\includegraphics[width=14.5cm]{../figs/04_supp/03_indicators/01_sst_a.png}
	\caption[Average daily sea surface temperature (SST) from 1980 to 2023 over coral reefs of countries and territories of the Pacific]{Average daily sea surface temperature (SST) from 1980 to 2023 over coral reefs of countries and territories of the Pacific. Purple lines represents the long-term SST average from 1980 to 2023 whereas the orange line represent the long-term trend.}
	\label{supp:fig-1a}
\end{figure}

\renewcommand{\thefigure}{S.\arabic{figure}.B}

\begin{figure}[h!]
	\centering
	\includegraphics[width=14.5cm]{../figs/04_supp/03_indicators/01_sst_b.png}
	\caption[Average daily sea surface temperature (SST) from 1980 to 2023 over coral reefs of countries and territories of the Pacific]{Average daily sea surface temperature (SST) from 1980 to 2023 over coral reefs of countries and territories of the Pacific. Purple lines represents the long-term SST average from 1980 to 2023 whereas the orange line represent the long-term trend.}
	\label{supp:fig-1b}
\end{figure}

\renewcommand{\thefigure}{S.\arabic{figure}}

\newpage
\clearpage

% BIBLIOGRAPHY
%%%%%%%%%%%%%%%%%%%%%%%%%%%%%%%%%%%%%%%%%%%%%%%%%%%%%%%%%%%%%%%%%%%%%%%%%%%%%%%%%%%%%%%%%%%%%%%%%%%%%%%%%%%%%

\begin{tikzpicture}[remember picture,overlay]
	\node[opacity=1,inner sep=0pt, anchor=north west] at ($(current page.north west)+(0cm,0cm)$){\includegraphics[width=\paperwidth]{../figs/00_misc/chapter.png}};
	\node[anchor=west] at ($(current page.north west)+(2.5cm,-0.75cm)$) {\textcolor{white}{{\scriptsize Hannes Klostermann. Coral reefs in Moorea, French Polynesia}}};
\end{tikzpicture}

\chaptermark{Bibliography}
\chapter*{Bibliography}
\addcontentsline{toc}{chapter}{Bibliography}

\begingroup
\def\chapter*#1{}
\vspace*{0cm}
\bibliographystyle{apa}
\bibliography{bibliography.bib}
\endgroup

% LAST PAGE
%%%%%%%%%%%%%%%%%%%%%%%%%%%%%%%%%%%%%%%%%%%%%%%%%%%%%%%%%%%%%%%%%%%%%%%%%%%%%%%%%%%%%%%%%%%%%%%%%%%%%%%%%%%%%

\newpage

\thispagestyle{empty}

\vspace*{22cm}

\begin{flushleft}
	{\footnotesize
		
		Document created using \LaTeX
			
		Compiled with MiKTeX version 23.12 
			
		On \today}
\end{flushleft}

\newpage

\thispagestyle{empty}

\begin{tikzpicture}[remember picture,overlay]
	\fill[anchor=west, secondcolor] ($(current page.north west)$) rectangle ($(current page.south east)$);
\end{tikzpicture}

\newpage

\thispagestyle{empty}

\begin{tikzpicture}[remember picture,overlay]
	% Main color
	\fill[anchor=west, firstcolor] ($(current page.north west)$) rectangle ($(current page.south east)$);
	% White borders
	\fill[anchor=north west, white] ($(current page.north west)+(-1cm,1cm)$) rectangle ($(current page.north west)+(21.5cm,-1.5cm)$);
	\fill[anchor=north west, white] ($(current page.north west)+(-1cm,-28.2cm)$) rectangle ($(current page.north west)+(21.5cm,-30cm)$);
	\fill[anchor=north west, white] ($(current page.north west)+(-1cm,1cm)$) rectangle ($(current page.north west)+(1.5cm,-30cm)$);
	\fill[anchor=north west, white] ($(current page.north west)+(19.5cm,1cm)$) rectangle ($(current page.north west)+(21.5cm,-30cm)$);
	% Text
	\node[align=center, text width=10cm] at ($(current page.center)+(0cm,-10cm)$){\textcolor{white}{{International Coral Reef Initiative}}};
	\node[align=center, text width=10cm] at ($(current page.center)+(0cm,-11cm)$){\textcolor{white}{{Global Coral Reef Monitoring Network}}};
\end{tikzpicture}

\end{document}